% (find-LATEX "2020-2-C3-aceleracao.tex")
% (defun c () (interactive) (find-LATEXsh "lualatex -record 2020-2-C3-aceleracao.tex" :end))
% (defun C () (interactive) (find-LATEXsh "lualatex 2020-2-C3-aceleracao.tex" "Success!!!"))
% (defun D () (interactive) (find-pdf-page      "~/LATEX/2020-2-C3-aceleracao.pdf"))
% (defun d () (interactive) (find-pdftools-page "~/LATEX/2020-2-C3-aceleracao.pdf"))
% (defun e () (interactive) (find-LATEX "2020-2-C3-aceleracao.tex"))
% (defun o () (interactive) (find-LATEX "2020-2-C3-aceleracao.tex"))
% (defun u () (interactive) (find-latex-upload-links "2020-2-C3-aceleracao"))
% (defun v () (interactive) (find-2a '(e) '(d)))
% (defun cv () (interactive) (C) (ee-kill-this-buffer) (v) (g))
% (defun d0 () (interactive) (find-ebuffer "2020-2-C3-aceleracao.pdf"))
% (find-pdf-page   "~/LATEX/2020-2-C3-aceleracao.pdf")
% (find-sh0 "cp -v  ~/LATEX/2020-2-C3-aceleracao.pdf /tmp/")
% (find-sh0 "cp -v  ~/LATEX/2020-2-C3-aceleracao.pdf /tmp/pen/")
%     (find-xournalpp "/tmp/2020-2-C3-aceleracao.pdf")
%   file:///home/edrx/LATEX/2020-2-C3-aceleracao.pdf
%               file:///tmp/2020-2-C3-aceleracao.pdf
%           file:///tmp/pen/2020-2-C3-aceleracao.pdf
% http://angg.twu.net/LATEX/2020-2-C3-aceleracao.pdf
% (find-LATEX "2019.mk")
% (find-CN-aula-links "2020-2-C3-aceleracao" "3" "c3m202acel" "c3m202ac")
%
% Video:
% (find-ssr-links "c3m202acel" "2020-2-C3-aceleracao")
% (code-video     "c3m202acelvideo" "$S/http/angg.twu.net/eev-videos/2020-2-C3-aceleracao.mp4")
% (find-c3m202acelvideo "0:00")
% (find-c3m202acelvideo "3:00")
% (find-c3m202acelvideo "3:40")
% (find-c3m202acelvideo "14:56" "interpretação geométrica de g e g'")

% (find-ssr-links "c3m202acel" "2020-2-C3-aceleracao-2")
% (code-video "c3m202acelvideo" "$S/http/angg.twu.net/eev-videos/2020-2-C3-aceleracao-2.mp4")
% (find-c3m202acelvideo "0:00")
% (find-c3m202acelvideo "1:18" "exercicio 2")
% (find-c3m202acelvideo "2:20" "com 0 vai ser bem fácil de calcular")
% (find-c3m202acelvideo "2:34" "motivação antes de prosseguir")
% (find-c3m202acelvideo "4:11" "aproximações pro seno")
% (find-c3m202acelvideo "4:55" "aos pouquinhos")
% (find-c3m202acelvideo "6:10" "quando a gente troca esse 0 daqui por 1")

% «.defs»	(to "defs")
% «.title»	(to "title")
% «.MRU»	(to "MRU")
% «.video»	(to "video")
% «.exercicio-2»	(to "exercicio-2")
%
% «.djvuize»	(to "djvuize")

\documentclass[oneside,12pt]{article}
\usepackage[colorlinks,citecolor=DarkRed,urlcolor=DarkRed]{hyperref} % (find-es "tex" "hyperref")
\usepackage{amsmath}
\usepackage{amsfonts}
\usepackage{amssymb}
\usepackage{pict2e}
\usepackage[x11names,svgnames]{xcolor} % (find-es "tex" "xcolor")
\usepackage{colorweb}                  % (find-es "tex" "colorweb")
%\usepackage{tikz}
%
% (find-dn6 "preamble6.lua" "preamble0")
%\usepackage{proof}   % For derivation trees ("%:" lines)
%\input diagxy        % For 2D diagrams ("%D" lines)
%\xyoption{curve}     % For the ".curve=" feature in 2D diagrams
%
\usepackage{edrx15}               % (find-LATEX "edrx15.sty")
\input edrxaccents.tex            % (find-LATEX "edrxaccents.tex")
\input edrxchars.tex              % (find-LATEX "edrxchars.tex")
\input edrxheadfoot.tex           % (find-LATEX "edrxheadfoot.tex")
\input edrxgac2.tex               % (find-LATEX "edrxgac2.tex")
%
%\usepackage[backend=biber,
%   style=alphabetic]{biblatex}            % (find-es "tex" "biber")
%\addbibresource{catsem-slides.bib}        % (find-LATEX "catsem-slides.bib")
%
% (find-es "tex" "geometry")
\usepackage[a6paper, landscape,
            top=1.5cm, bottom=.25cm, left=1cm, right=1cm, includefoot
           ]{geometry}
%
\begin{document}

%\catcode`\^^J=10
%\directlua{dofile "dednat6load.lua"}  % (find-LATEX "dednat6load.lua")

% %L dofile "edrxtikz.lua"  -- (find-LATEX "edrxtikz.lua")
% %L dofile "edrxpict.lua"  -- (find-LATEX "edrxpict.lua")
% \pu

% «defs»  (to ".defs")
% (find-LATEX "edrx15.sty" "colors-2019")
\long\def\ColorRed   #1{{\color{Red1}#1}}
\long\def\ColorViolet#1{{\color{MagentaVioletLight}#1}}
\long\def\ColorViolet#1{{\color{Violet!50!black}#1}}
\long\def\ColorGreen #1{{\color{SpringDarkHard}#1}}
\long\def\ColorGreen #1{{\color{SpringGreenDark}#1}}
\long\def\ColorGreen #1{{\color{SpringGreen4}#1}}
\long\def\ColorGray  #1{{\color{GrayLight}#1}}
\long\def\ColorGray  #1{{\color{black!30!white}#1}}
\long\def\ColorBrown #1{{\color{Brown}#1}}
\long\def\ColorBrown #1{{\color{brown}#1}}

\long\def\ColorShort #1{{\color{SpringGreen4}#1}}
\long\def\ColorLong  #1{{\color{Red1}#1}}

\def\frown{\ensuremath{{=}{(}}}
\def\True {\mathbf{V}}
\def\False{\mathbf{F}}

\def\drafturl{http://angg.twu.net/LATEX/2020-2-C3.pdf}
\def\drafturl{http://angg.twu.net/2020.2-C3.html}
\def\draftfooter{\tiny \href{\drafturl}{\jobname{}} \ColorBrown{\shorttoday{} \hours}}



%  _____ _ _   _                               
% |_   _(_) |_| | ___   _ __   __ _  __ _  ___ 
%   | | | | __| |/ _ \ | '_ \ / _` |/ _` |/ _ \
%   | | | | |_| |  __/ | |_) | (_| | (_| |  __/
%   |_| |_|\__|_|\___| | .__/ \__,_|\__, |\___|
%                      |_|          |___/      
%
% «title»  (to ".title")
% (c3m202acelp 1 "title")
% (c3m202acel    "title")

\thispagestyle{empty}

\begin{center}

\vspace*{1.2cm}

{\bf \Large Cálculo 3 - 2020.2}

\bsk

Aula 3: o vetor aceleração

\bsk

Eduardo Ochs - RCN/PURO/UFF

\url{http://angg.twu.net/2020.2-C3.html}

\end{center}

\newpage

% «MRU»  (to ".MRU")
% (c3m202acelp 2 "MRU")
% (c3m202acel    "MRU")

Em Física a aceleração de uma partícula é ``o quanto a velocidade dela
varia com o tempo''; a posição é dada em metros, a velocidade em $m/s$
e a aceleração em $m/s^2$ --- e a gente começa a entender aceleração
aprendendo a visualizar e fazer as contas em dois casos:

\begin{enumerate}

\item Movimento uniformemente acelerado na vertical. Por exemplo:
  ``considere uma partícula que se move na vertical em M.U.A., com
  aceleração igual a $2 \frac{m}{s^2}$ para cima. Digamos que em
  $t=3s$ ela está no ponto $y=4m$ e a velocidade dela é $5
  \frac{m}{s}$ para cima''...

\item Movimento uniformemente acelerado no plano. Por exemplo:
  ``considere uma partícula que se move no plano $(x,y)$ em M.U.A.,
  com aceleração igual a $\VEC{2,3} \frac{m}{s^2}$. Digamos que em
  $t=4s$ a velocidade dela é $\VEC{5,6} \frac{m}{s}$''...

\end{enumerate}

\newpage

Em cálculo 3 nós (geralmente) não vamos usar unidades de medida como
metros e segundos. Daqui a algumas aulas nós vamos ver como
praticamente todas as contas de Cálculo 3 podem ser ``tipadas'' neste
sentido aqui:

{\footnotesize

% (c3m201derpsp 5 "tipos")
% (c3m201derps    "tipos")
\url{http://angg.twu.net/LATEX/2020-1-C3-derivs-parciais.pdf\#page=5}

}


\bsk
\bsk

{\bf Exercício 1}

O Bortolossi fala {\sl explicitamente} do vetor aceleração em bem
poucos lugares do livro dele, mas implicitamente em vários lugares.
Estou gravando um vídeo sobre isso... se você já terminou os
exercícios da aula passada faça o exercício 15 do Bortolossi, que está
na página 217, que está no capítulo 6.

% \newpage

% «video»  (to ".video")

% {\bf Algumas coisas do vídeo}

% (find-c3m202acelvideo "3:00")
% (find-c3m202acelvideo "3:40")



\newpage

% «exercicio-2»  (to ".exercicio-2")
% (c3m202acelp 4 "exercicio-2")
% (c3m202acel    "exercicio-2")

{\bf Exercício 2}

\ssk

Em cada um dos casos abaixo calcule $g(x)$ e $g'(x)$ para $x=b$,
$x=b+1$ e $x=b-1$ e use isto pra desenhar três pontos do gráfico da
$g$ e a inclinação do gráfico da $g$ nestes três pontos, e aí use isto
pra desenhar o gráfico da parábola $y=g(x)$ em torno de $x=b$.

\msk

a) $b=3$ e $g(x) = 2 + 0·(x-b) + 0·(x-b)^2$

b) $b=3$ e $g(x) = 2 + 0·(x-b) + 1·(x-b)^2$

c) $b=3$ e $g(x) = 2 + 0·(x-b) + -1·(x-b)^2$

\ssk

d) $b=4$ e $g(x) = 2 + 0·(x-b) + 0·(x-b)^2$

e) $b=4$ e $g(x) = 2 + 0·(x-b) + 1·(x-b)^2$

f) $b=4$ e $g(x) = 2 + 0·(x-b) + -1·(x-b)^2$

\ssk

g) $b=4$ e $g(x) = 1 + 0·(x-b) + 0·(x-b)^2$

h) $b=4$ e $g(x) = 1 + 0·(x-b) + 1·(x-b)^2$

i) $b=4$ e $g(x) = 1 + 0·(x-b) + -1·(x-b)^2$

\newpage

j) $b=4$ e $g(x) = 1 + 1·(x-b) + 0·(x-b)^2$

k) $b=4$ e $g(x) = 1 + 1·(x-b) + 1·(x-b)^2$

l) $b=4$ e $g(x) = 1 + 1·(x-b) + -1·(x-b)^2$

\ssk

m) $b=4$ e $g(x) = 1 + -1·(x-b) + 0·(x-b)^2$

n) $b=4$ e $g(x) = 1 + -1·(x-b) + 1·(x-b)^2$

o) $b=4$ e $g(x) = 1 + -1·(x-b) + -1·(x-b)^2$

\ssk

p) $b=4$ e $g(x) = 1 + -1·(x-b) + 0·(x-b)^2$

q) $b=4$ e $g(x) = 1 + -1·(x-b) + \frac12·(x-b)^2$

r) $b=4$ e $g(x) = 1 + -1·(x-b) + -\frac12·(x-b)^2$





% Pra gente uma trajetória em $\R^2$ é uma função de $\R$ em $\R^2$,
% onde esse


% (find-bortolossi6page (+ -186 187) "6. Curvas parametrizadas, transformações lineares...")
% (find-bortolossi6page (+ -186 188)   "traço")
% (find-bortolossi6page (+ -186 195)   "Figura 6.6. Traço da hélice")
% (find-bortolossi6page (+ -186 197) "6.2 O vetor tangente a uma curva parametrizada")
% (find-bortolossi6page (+ -186 199)   "limite de vetores secantes")
% (find-bortolossi6page (+ -186 215)   "A ciclóide")
% (find-bortolossi6page (+ -186 217)   "O vetor aceleração")



\newpage

\phantom{a}

\newpage

\phantom{a}

\newpage

%\printbibliography

\GenericWarning{Success:}{Success!!!}  % Used by `M-x cv'

\end{document}

%  ____  _             _         
% |  _ \(_)_   ___   _(_)_______ 
% | | | | \ \ / / | | | |_  / _ \
% | |_| | |\ V /| |_| | |/ /  __/
% |____// | \_/  \__,_|_/___\___|
%     |__/                       
%
% «djvuize»  (to ".djvuize")
% (find-LATEXgrep "grep --color -nH --null -e djvuize 2020-1*.tex")

 (eepitch-shell)
 (eepitch-kill)
 (eepitch-shell)
# (find-fline "~/2020.2-C3/")
# (find-fline "~/LATEX/2020-2-C3/")
# (find-fline "~/bin/djvuize")

cd /tmp/
for i in *.jpg; do echo f $(basename $i .jpg); done

f () { rm -fv $1.png $1.pdf; djvuize $1.pdf }
f () { rm -fv $1.png $1.pdf; djvuize WHITEBOARDOPTS="-m 1.0" $1.pdf; xpdf $1.pdf }
f () { rm -fv $1.png $1.pdf; djvuize WHITEBOARDOPTS="-m 0.5" $1.pdf; xpdf $1.pdf }
f () { rm -fv $1.png $1.pdf; djvuize WHITEBOARDOPTS="-m 0.25" $1.pdf; xpdf $1.pdf }
f () { cp -fv $1.png $1.pdf       ~/2020.2-C3/
       cp -fv        $1.pdf ~/LATEX/2020-2-C3/
       cat <<%%%
% (find-latexscan-links "C3" "$1")
%%%
}

f 20201213_area_em_funcao_de_theta
f 20201213_area_em_funcao_de_x
f 20201213_area_fatias_pizza



%  __  __       _        
% |  \/  | __ _| | _____ 
% | |\/| |/ _` | |/ / _ \
% | |  | | (_| |   <  __/
% |_|  |_|\__,_|_|\_\___|
%                        
% <make>

 (eepitch-shell)
 (eepitch-kill)
 (eepitch-shell)
# (find-LATEXfile "2019planar-has-1.mk")
make -f 2019.mk STEM=2020-2-C3-aceleracao veryclean
make -f 2019.mk STEM=2020-2-C3-aceleracao pdf

% Local Variables:
% coding: utf-8-unix
% ee-tla: "c3m202acel"
% End:
