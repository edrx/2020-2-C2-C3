% (find-LATEX "2020-2-C3-plano-tang.tex")
% (defun c () (interactive) (find-LATEXsh "lualatex -record 2020-2-C3-plano-tang.tex" :end))
% (defun C () (interactive) (find-LATEXsh "lualatex 2020-2-C3-plano-tang.tex" "Success!!!"))
% (defun D () (interactive) (find-pdf-page      "~/LATEX/2020-2-C3-plano-tang.pdf"))
% (defun d () (interactive) (find-pdftools-page "~/LATEX/2020-2-C3-plano-tang.pdf"))
% (defun e () (interactive) (find-LATEX "2020-2-C3-plano-tang.tex"))
% (defun o () (interactive) (find-LATEX "2020-2-C3-plano-tang.tex"))
% (defun u () (interactive) (find-latex-upload-links "2020-2-C3-plano-tang"))
% (defun v () (interactive) (find-2a '(e) '(d)))
% (defun d0 () (interactive) (find-ebuffer "2020-2-C3-plano-tang.pdf"))
% (defun cv () (interactive) (C) (ee-kill-this-buffer) (v) (g))
%          (code-eec-LATEX "2020-2-C3-plano-tang")
% (find-pdf-page   "~/LATEX/2020-2-C3-plano-tang.pdf")
% (find-sh0 "cp -v  ~/LATEX/2020-2-C3-plano-tang.pdf /tmp/")
% (find-sh0 "cp -v  ~/LATEX/2020-2-C3-plano-tang.pdf /tmp/pen/")
%     (find-xournalpp "/tmp/2020-2-C3-plano-tang.pdf")
%   file:///home/edrx/LATEX/2020-2-C3-plano-tang.pdf
%               file:///tmp/2020-2-C3-plano-tang.pdf
%           file:///tmp/pen/2020-2-C3-plano-tang.pdf
% http://angg.twu.net/LATEX/2020-2-C3-plano-tang.pdf
% (find-LATEX "2019.mk")
% (find-CN-aula-links "2020-2-C3-plano-tang" "3" "c3m202planotang" "c3pt")
%
% Video:
% (find-ssr-links "c3m202planotang" "2020-2-C3-plano-tang")
% (code-video     "c3m202planotangvideo" "$S/http/angg.twu.net/eev-videos/2020-2-C3-plano-tang.mp4")
% (find-c3m202planotangvideo "0:00")

% «.defs»	(to "defs")
% «.title»	(to "title")
%
% «.djvuize»	(to "djvuize")

\documentclass[oneside,12pt]{article}
\usepackage[colorlinks,citecolor=DarkRed,urlcolor=DarkRed]{hyperref} % (find-es "tex" "hyperref")
\usepackage{amsmath}
\usepackage{amsfonts}
\usepackage{amssymb}
\usepackage{pict2e}
\usepackage[x11names,svgnames]{xcolor} % (find-es "tex" "xcolor")
\usepackage{colorweb}                  % (find-es "tex" "colorweb")
%\usepackage{tikz}
%
% (find-dn6 "preamble6.lua" "preamble0")
%\usepackage{proof}   % For derivation trees ("%:" lines)
%\input diagxy        % For 2D diagrams ("%D" lines)
%\xyoption{curve}     % For the ".curve=" feature in 2D diagrams
%
\usepackage{edrx15}               % (find-LATEX "edrx15.sty")
\input edrxaccents.tex            % (find-LATEX "edrxaccents.tex")
\input edrxchars.tex              % (find-LATEX "edrxchars.tex")
\input edrxheadfoot.tex           % (find-LATEX "edrxheadfoot.tex")
\input edrxgac2.tex               % (find-LATEX "edrxgac2.tex")
%
%\usepackage[backend=biber,
%   style=alphabetic]{biblatex}            % (find-es "tex" "biber")
%\addbibresource{catsem-slides.bib}        % (find-LATEX "catsem-slides.bib")
%
% (find-es "tex" "geometry")
\usepackage[a6paper, landscape,
            top=1.5cm, bottom=.25cm, left=1cm, right=1cm, includefoot
           ]{geometry}
%
\begin{document}

%\catcode`\^^J=10
%\directlua{dofile "dednat6load.lua"}  % (find-LATEX "dednat6load.lua")

% %L dofile "edrxtikz.lua"  -- (find-LATEX "edrxtikz.lua")
% %L dofile "edrxpict.lua"  -- (find-LATEX "edrxpict.lua")
% \pu

% «defs»  (to ".defs")
% (find-LATEX "edrx15.sty" "colors-2019")
\long\def\ColorRed   #1{{\color{Red1}#1}}
\long\def\ColorViolet#1{{\color{MagentaVioletLight}#1}}
\long\def\ColorViolet#1{{\color{Violet!50!black}#1}}
\long\def\ColorGreen #1{{\color{SpringDarkHard}#1}}
\long\def\ColorGreen #1{{\color{SpringGreenDark}#1}}
\long\def\ColorGreen #1{{\color{SpringGreen4}#1}}
\long\def\ColorGray  #1{{\color{GrayLight}#1}}
\long\def\ColorGray  #1{{\color{black!30!white}#1}}
\long\def\ColorBrown #1{{\color{Brown}#1}}
\long\def\ColorBrown #1{{\color{brown}#1}}
\long\def\ColorOrange#1{{\color{orange}#1}}

\long\def\ColorShort #1{{\color{SpringGreen4}#1}}
\long\def\ColorLong  #1{{\color{Red1}#1}}

\def\frown{\ensuremath{{=}{(}}}
\def\True {\mathbf{V}}
\def\False{\mathbf{F}}
\def\D    {\displaystyle}

\def\drafturl{http://angg.twu.net/LATEX/2020-2-C3.pdf}
\def\drafturl{http://angg.twu.net/2020.2-C3.html}
\def\draftfooter{\tiny \href{\drafturl}{\jobname{}} \ColorBrown{\shorttoday{} \hours}}



%  _____ _ _   _                               
% |_   _(_) |_| | ___   _ __   __ _  __ _  ___ 
%   | | | | __| |/ _ \ | '_ \ / _` |/ _` |/ _ \
%   | | | | |_| |  __/ | |_) | (_| | (_| |  __/
%   |_| |_|\__|_|\___| | .__/ \__,_|\__, |\___|
%                      |_|          |___/      
%
% «title»  (to ".title")
% (c3m202planotangp 1 "title")
% (c3m202planotang    "title")

\thispagestyle{empty}

\begin{center}

\vspace*{1.2cm}

{\bf \Large Cálculo 3 - 2020.2}

\bsk

Aula ??: o plano tangente

\bsk

Eduardo Ochs - RCN/PURO/UFF

\url{http://angg.twu.net/2020.2-C3.html}

\end{center}

\newpage

Os exercícios de hoje são pra ajudar todo mundo a entender

e visualizar as idéias principais do capítulo 7 do

Bortolossi... antes de começar dê uma olhada nele!




\newpage

{\bf Plano tangente: definição formal}

Digamos que a superfície $S$ seja dada por:
%
% (c3m202rcadeia1p 13 "exercicio-5")
% (c3m202rcadeia1     "exercicio-5")
%
$$S = \setofxyzst{z = F(x,y)},$$

como na aula passada a partir do exercício 5:

\ssk

% http://angg.twu.net/LATEX/2020-2-C3-rcadeia1.pdf
\url{http://angg.twu.net/LATEX/2020-2-C3-rcadeia1.pdf\#page=11}

\ssk

O plano tangente à superfície $S$ no ponto $(x_0,y_0,F(x_0,y_0))$ ---

ou no ponto $(x_0,y_0)$, se a gente usar a gambiarra que nos

permite rededuzir a coordenada $z$ desse ponto a partir das

coordenadas $x$ e $y$ dele --- pode ser definido como um

plano parameterizado assim:

\newpage

{\bf Plano tangente: definição formal (2)}

\msk

$$\begin{array}{rcl}
  \vv &=& \VEC{1,0,\frac{∂z}{∂x}} \\
    % &=& \VEC{1,0,\frac{∂z}{∂x}(x_0,y_0)} \\
      &=& \VEC{1,0,F_x} \\
      &=& \VEC{1,0,F_x(x_0,y_0)} \\
  \ww &=& \VEC{0,1,\frac{∂z}{∂y}} \\
    % &=& \VEC{0,1,\frac{∂z}{∂y}(x_0,y_0)} \\
      &=& \VEC{0,1,F_y} \\
      &=& \VEC{0,1,F_y(x_0,y_0)} \\
  P(t,u) &=& (x_0,y_0,z_0) + t\vv + u\ww \\
  π &=& \setofst{P(t,u)}{t,u∈\R} \\
    &=& \setofst{(x_0,y_0,z_0) + t\VEC{1,0,\frac{∂z}{∂x}} + u\VEC{0,1,\frac{∂z}{∂y}}}{t,u∈\R} \\
  \end{array}
$$

\newpage

Nós vamos começar entendendo isto em casos nos quais a

superfície $S$ é um plano... mas primeiro vamos fazer alguns

exercícios de desenhar os diagramas de numerozinhos e

algumas curvas de nível de planos.


\msk

{\bf Exercício 1.}

Em cada um dos casos abaixo a função $F$ vai ser definida por
%
$$F(x,y) = a + b·(x-x_0) + c·(y-y_0).$$

Represente graficamente em 3D os pontos da superfície $S$

associados aos pontos

$(x_0,y_0), (x_0+1,y_0), (x_0,y_0+1), (x_0+1,y_0+1)$.

Além disso faça o diagrama de numerozinhos da $F$ e

desenhe no plano $(x,y)$ pelo menos duas curvas de

nível da função $F$.


\newpage

{\bf Exercício 1 (cont.)}

a) $(x_0,y_0) = (4,2)$, $a=2$, $b=1$, $c=0$.

b) $(x_0,y_0) = (4,2)$, $a=2$, $b=0$, $c=1$.

c) $(x_0,y_0) = (4,2)$, $a=2$, $b=1$, $c=1$.

d) $(x_0,y_0) = (4,2)$, $a=2$, $b=2$, $c=-1$.

e) $(x_0,y_0) = (3,3)$, $a=2$, $b=2$, $c=-1$.







%\printbibliography

\GenericWarning{Success:}{Success!!!}  % Used by `M-x cv'

\end{document}

%  ____  _             _         
% |  _ \(_)_   ___   _(_)_______ 
% | | | | \ \ / / | | | |_  / _ \
% | |_| | |\ V /| |_| | |/ /  __/
% |____// | \_/  \__,_|_/___\___|
%     |__/                       
%
% «djvuize»  (to ".djvuize")
% (find-LATEXgrep "grep --color -nH --null -e djvuize 2020-1*.tex")

 (eepitch-shell)
 (eepitch-kill)
 (eepitch-shell)
# (find-fline "~/2020.2-C3/")
# (find-fline "~/LATEX/2020-2-C3/")
# (find-fline "~/bin/djvuize")

cd /tmp/
for i in *.jpg; do echo f $(basename $i .jpg); done

f () { rm -fv $1.png $1.pdf; djvuize $1.pdf }
f () { rm -fv $1.png $1.pdf; djvuize WHITEBOARDOPTS="-m 1.0" $1.pdf; xpdf $1.pdf }
f () { rm -fv $1.png $1.pdf; djvuize WHITEBOARDOPTS="-m 0.5" $1.pdf; xpdf $1.pdf }
f () { rm -fv $1.png $1.pdf; djvuize WHITEBOARDOPTS="-m 0.25" $1.pdf; xpdf $1.pdf }
f () { cp -fv $1.png $1.pdf       ~/2020.2-C3/
       cp -fv        $1.pdf ~/LATEX/2020-2-C3/
       cat <<%%%
% (find-latexscan-links "C3" "$1")
%%%
}

f 20201213_area_em_funcao_de_theta
f 20201213_area_em_funcao_de_x
f 20201213_area_fatias_pizza



%  __  __       _        
% |  \/  | __ _| | _____ 
% | |\/| |/ _` | |/ / _ \
% | |  | | (_| |   <  __/
% |_|  |_|\__,_|_|\_\___|
%                        
% <make>

 (eepitch-shell)
 (eepitch-kill)
 (eepitch-shell)
# (find-LATEXfile "2019planar-has-1.mk")
make -f 2019.mk STEM=2020-2-C3-plano-tang veryclean
make -f 2019.mk STEM=2020-2-C3-plano-tang pdf

% Local Variables:
% coding: utf-8-unix
% ee-tla: "c3m202planotang"
% End:
