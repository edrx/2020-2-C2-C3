% (find-LATEX "2020-2-C2-escadas.tex")
% (defun c () (interactive) (find-LATEXsh "lualatex -record 2020-2-C2-escadas.tex" :end))
% (defun C () (interactive) (find-LATEXsh "lualatex 2020-2-C2-escadas.tex" "Success!!!"))
% (defun D () (interactive) (find-pdf-page      "~/LATEX/2020-2-C2-escadas.pdf"))
% (defun d () (interactive) (find-pdftools-page "~/LATEX/2020-2-C2-escadas.pdf"))
% (defun e () (interactive) (find-LATEX "2020-2-C2-escadas.tex"))
% (defun o () (interactive) (find-LATEX "2020-2-C2-escadas.tex"))
% (defun u () (interactive) (find-latex-upload-links "2020-2-C2-escadas"))
% (defun v () (interactive) (find-2a '(e) '(d)))
% (defun d0 () (interactive) (find-ebuffer "2020-2-C2-escadas.pdf"))
% (defun cv () (interactive) (C) (ee-kill-this-buffer) (v) (g))
%          (code-eec-LATEX "2020-2-C2-escadas")
% (find-pdf-page   "~/LATEX/2020-2-C2-escadas.pdf")
% (find-sh0 "cp -v  ~/LATEX/2020-2-C2-escadas.pdf /tmp/")
% (find-sh0 "cp -v  ~/LATEX/2020-2-C2-escadas.pdf /tmp/pen/")
%     (find-xournalpp "/tmp/2020-2-C2-escadas.pdf")
%   file:///home/edrx/LATEX/2020-2-C2-escadas.pdf
%               file:///tmp/2020-2-C2-escadas.pdf
%           file:///tmp/pen/2020-2-C2-escadas.pdf
% http://angg.twu.net/LATEX/2020-2-C2-escadas.pdf
% (find-LATEX "2019.mk")
% (find-CN-aula-links "2020-2-C2-escadas" "2" "c2m202escadas" "c28")
%
% Video:
% (find-ssr-links "c2m202escadas" "2020-2-C2-escadas")
% (code-video     "c2m202escadasvideo" "$S/http/angg.twu.net/eev-videos/2020-2-C2-escadas.mp4")
% (find-c2m202escadasvideo "0:00")

% (find-ssr-links "c2m202escadasb" "2020-2-C2-escadas-b")

% «.defs»		(to "defs")
% «.title»		(to "title")
% «.funcoes-escada»	(to "funcoes-escada")
% «.exercicio-1»	(to "exercicio-1")
% «.exercicio-2»	(to "exercicio-2")
% «.exercicio-4»	(to "exercicio-4")
% «.exercicio-5»	(to "exercicio-5")
% «.exercicio-6»	(to "exercicio-6")
% «.exercicio-7»	(to "exercicio-7")
% «.primitivas-como-usar»	(to "primitivas-como-usar")
%
% «.djvuize»	(to "djvuize")

\documentclass[oneside,12pt]{article}
\usepackage[colorlinks,citecolor=DarkRed,urlcolor=DarkRed]{hyperref} % (find-es "tex" "hyperref")
\usepackage{amsmath}
\usepackage{amsfonts}
\usepackage{amssymb}
\usepackage{pict2e}
\usepackage[x11names,svgnames]{xcolor} % (find-es "tex" "xcolor")
\usepackage{colorweb}                  % (find-es "tex" "colorweb")
%\usepackage{tikz}
%
% (find-dn6 "preamble6.lua" "preamble0")
\usepackage{proof}   % For derivation trees ("%:" lines)
\input diagxy        % For 2D diagrams ("%D" lines)
\xyoption{curve}     % For the ".curve=" feature in 2D diagrams
%
\usepackage{edrx15}               % (find-LATEX "edrx15.sty")
\input edrxaccents.tex            % (find-LATEX "edrxaccents.tex")
\input edrxchars.tex              % (find-LATEX "edrxchars.tex")
\input edrxheadfoot.tex           % (find-LATEX "edrxheadfoot.tex")
\input edrxgac2.tex               % (find-LATEX "edrxgac2.tex")
%
%\usepackage[backend=biber,
%   style=alphabetic]{biblatex}            % (find-es "tex" "biber")
%\addbibresource{catsem-slides.bib}        % (find-LATEX "catsem-slides.bib")
%
% (find-es "tex" "geometry")
\usepackage[a6paper, landscape,
            top=1.5cm, bottom=.25cm, left=1cm, right=1cm, includefoot
           ]{geometry}
%
\begin{document}

\catcode`\^^J=10
\directlua{dofile "dednat6load.lua"}  % (find-LATEX "dednat6load.lua")

%L dofile "edrxtikz.lua"  -- (find-LATEX "edrxtikz.lua")
%L dofile "edrxpict.lua"  -- (find-LATEX "edrxpict.lua")
\pu

% «defs»  (to ".defs")
% (find-LATEX "edrx15.sty" "colors-2019")
\long\def\ColorRed   #1{{\color{Red1}#1}}
\long\def\ColorViolet#1{{\color{MagentaVioletLight}#1}}
\long\def\ColorViolet#1{{\color{Violet!50!black}#1}}
\long\def\ColorGreen #1{{\color{SpringDarkHard}#1}}
\long\def\ColorGreen #1{{\color{SpringGreenDark}#1}}
\long\def\ColorGreen #1{{\color{SpringGreen4}#1}}
\long\def\ColorGray  #1{{\color{GrayLight}#1}}
\long\def\ColorGray  #1{{\color{black!30!white}#1}}
\long\def\ColorBrown #1{{\color{Brown}#1}}
\long\def\ColorBrown #1{{\color{brown}#1}}
\long\def\ColorOrange#1{{\color{orange}#1}}

\long\def\ColorShort #1{{\color{SpringGreen4}#1}}
\long\def\ColorLong  #1{{\color{Red1}#1}}

\def\frown{\ensuremath{{=}{(}}}
\def\True {\mathbf{V}}
\def\False{\mathbf{F}}
\def\D    {\displaystyle}

\def\drafturl{http://angg.twu.net/LATEX/2020-2-C2.pdf}
\def\drafturl{http://angg.twu.net/2020.2-C2.html}
\def\draftfooter{\tiny \href{\drafturl}{\jobname{}} \ColorBrown{\shorttoday{} \hours}}

% (find-LATEX "2020-2-C2-somas-2.tex" "intover-intunder")

\def\Intover #1#2{\overline {∫}_{#1}#2\,dx}
\def\Intunder#1#2{\underline{∫}_{#1}#2\,dx}
\def\Intoverunder#1#2{\Intover{#1}{#2} - \Intunder{#1}{#2}}
%
\def\Intxover #1#2#3{\overline {∫}_{x=#1}^{x=#2}#3\,dx}
\def\Intxunder#1#2#3{\underline{∫}_{x=#1}^{x=#2}#3\,dx}



%  _____ _ _   _                               
% |_   _(_) |_| | ___   _ __   __ _  __ _  ___ 
%   | | | | __| |/ _ \ | '_ \ / _` |/ _` |/ _ \
%   | | | | |_| |  __/ | |_) | (_| | (_| |  __/
%   |_| |_|\__|_|\___| | .__/ \__,_|\__, |\___|
%                      |_|          |___/      
%
% «title»  (to ".title")
% (c2m202escadasp 1 "title")
% (c2m202escadas    "title")

\thispagestyle{empty}

\begin{center}

\vspace*{1.2cm}

{\bf \Large Cálculo 2 - 2020.2}

\bsk

Aula 8: integrais de funções escada

\bsk

Eduardo Ochs - RCN/PURO/UFF

\url{http://angg.twu.net/2020.2-C2.html}

\end{center}

\newpage

Dê uma olhada nas propriedades da integral que o Pierluigi Beneveri
demonstra (!!!) nas páginas 5 a 8 das notas dele:

\msk

% (find-books "__analysis/__analysis.el" "beneveri")
%
{\footnotesize
\url{https://www.ime.usp.br/~pluigi/registro-MAT121-15.pdf}

\url{http://angg.twu.net/2020.2-C2/pierluigi_beneveri_MAT121-15.pdf}
}

\msk

As demonstrações {\sl formais}, como ele faz, com estimativas e
somatórios, não nos interessam neste curso... mas todas as
demonstrações dele podem ser ``traduzidas'' pra argumentos visuais
como os que você deve ter entendido fazendo os exercícios da aula
passada.

\msk

Nos exercícios de hoje nós vamos usar principalmente o Exercício 18 da
página 5 das notas do Pierluigi e a Proposição 8/Propriedade 4 da
página 7.

% (find-pierluigipage  5    "Exercício 18")
% (find-pierluigitext  5    "Exercício 18")
% (find-pierluigipage  7    "Proposição 8 (Propriedade 4")
% (find-pierluigitext  7    "Proposição 8 (Propriedade 4")
% (find-pierluigipage  8    "Definição 9")
% (find-pierluigitext  8    "Definição 9")

\ColorRed{Leia também a Definição 9 na página 8}, principalmente os
comentários (1) e (2)... hoje nós vamos começar a ter que lidar com
``áreas negativas''!


\newpage

% «funcoes-escada»  (to ".funcoes-escada")
% (c2m202escadasp 3 "funcoes-escada")
% (c2m202escadas    "funcoes-escada")
% (c2m201escadasp 2 "funcoes-escada")
% (c2m201escadas    "funcoes-escada")

{\bf Funções escada}

Uma {\bf função escada} é uma cujo gráfico é composto por um número
finito de segmentos horizontais e um número finito -- talvez zero --
de pontos isolados. Por exemplo:
%
$$
 f(x) \;\; = \;\;
 \vcenter{\hbox{%
 \unitlength=10pt
 \beginpicture(0,-2)(6,5)
   \pictgrid%
   \pictpiecewise{(0,3)--(1,3)o (1,4)c (1,2)o--(3,2)c (3,-1)o--(6,-1)}%
   \pictaxes%
 \end{picture}%
 }}
$$

\vspace{-20pt}

% «exercicio-1»  (to ".exercicio-1")
% (c2m202escadasp 3 "exercicio-1")
% (c2m202escadas    "exercicio-1")

{\bf Exercício 1.}

Calcule:

a) $\Intx01 {f(x)}$

b) $\Intx13 {f(x)}$

c) $\Intx34 {f(x)}$

d) $\Intx04 {f(x)}$ \quad (veja o exercício 3)


\newpage

% «exercicio-2»  (to ".exercicio-2")
% (c2m202escadasp 4 "exercicio-2")
% (c2m202escadas    "exercicio-2")

{\bf Exercício 2.}

\ssk

Agora vamos tentar integrar a $f(x)$ da página anterior

usando as definições dos slides que usamos nas últimas aulas...

Seja $[a,b]$ o intervalo $[0,3]$.

Seja $\{P_0, P_1, P_2, \ldots\}$ a nossa sequência preferida

de partições do intervalo $[a,b]$.

\ssk

a) Quantos intervalos tem $P_{10}$?

b) Quantos pontos tem $P_{10}$?

c) Qual é a largura de cada intervalo de $P_{10}$?

d) Represente graficamente $\Intoverunder{P_{10}}{f(x)}$.

e) A resposta do item anterior é um retângulo. Qual é a sua base?

\phantom{e) }Qual é a sua altura? Qual é a sua área?

f) Calcule $\Intoverunder{P_{10}}{f(x)}$.

g) Calcule $\Intoverunder{P_{1000}}{f(x)}$.



\newpage

{\bf Exercício 3.}

Seja $f$ esta função (a mesma do slide 3):

$$
 \unitlength=7.5pt
 %
 f(x) \;\; = \;\;
 \vcenter{\hbox{%
 \beginpicture(0,-2)(6,5)
   \pictgrid%
   \pictpiecewise{(0,3)--(1,3)o (1,4)c (1,2)o--(3,2)c (3,-1)o--(6,-1)}%
   \pictaxes%
 \end{picture}%
 }}
$$

\msk

Dá pra calcular $\Intx{0}{4}{f(x)}$ assim:
%
$$\begin{array}{rcl}
  \Intx{0}{4}{f(x)} &=& \Intx{0}{1}{f(x)} + \Intx{1}{3}{f(x)} + \Intx{3}{4}{f(x)} \\
                    &=& 3·(1-0) + 2·(3-1) + (-1)·(4-3) \\
                    &=& 3 + 4 - 1 \;\; = \;\; 6 \\
  \end{array}
$$

Descubra quais propriedades/proposições/exercícios/etc

do Pierluigi nós usamos em cada `$=$' acima.

\newpage

% «exercicio-4»  (to ".exercicio-4")
% (c2m202escadasp 6 "exercicio-4")
% (c2m202escadas    "exercicio-4")

{\bf Exercício 4.}

Sejam $f$ e $g$ estas funções:

$$
 \unitlength=7.5pt
 %
 f(x) \;\; = \;\;
 \vcenter{\hbox{%
 \beginpicture(0,-2)(6,5)
   \pictgrid%
   \pictpiecewise{(0,3)--(1,3)o (1,4)c (1,2)o--(3,2)c (3,-1)o--(6,-1)}%
   \pictaxes%
 \end{picture}%
 }}
 %
 \qquad
 %
 g(x) \;\; = \;\;
 \vcenter{\hbox{%
 \beginpicture(0,-2)(6,5)
   \pictgrid%
   \pictpiecewise{(0,3)--(1,3)o (1,1)c (1,2)o--(3,2)c (3,-1)o--(6,-1)}%
   \pictaxes%
 \end{picture}%
 }}
 %
$$

\msk

Elas são integráveis no intervalo $[0,4]$

e só diferem no ponto $x=1$, $1∈[0,4]$...

\ColorRed{Então} $\Intx{0}{4}{f(x)} = \Intx{0}{4}{g(x)}$.

\msk

Acho que o Pierluigi não explica explicitamente porque

esse ``então'' é verdade. Vamos ver isto passo a passo.

\msk

a) Seja $h(x) = f(x) - g(x)$. \;\; \ColorGray{(Fica implícito que é ``$∀x$''.)}

Faça o gráfico da $h(x)$.

\newpage

{\bf Exercício 4 (cont.)}

\ssk

b) Calcule $\Intoverunder{P_{10}}{h(x)}$.

c) Calcule $\Intoverunder{P_{1000}}{h(x)}$.

d) Conclua que $\Intx{0}{4}{h(x)} = 0$.

\bsk

Dá pra provar que $\Intx{0}{4}{f(x)} = \Intx{0}{4}{g(x)}$ assim:
%
$$\begin{array}{rcl}
  \Intx{0}{4}{f(x)} &=& \Intx{0}{4}{g(x) + h(x)} \\
                    &=& \Intx{0}{4}{g(x)} + \Intx{0}{4}{h(x)} \\
                    &=& \Intx{0}{4}{g(x)} + 0 \\
                    &=& \Intx{0}{4}{g(x)} \\
  \end{array}
$$

e) Descubra quais propriedades/proposições/exercícios/etc

do Pierluigi nós usamos em cada `$=$' acima.


\newpage

Até agora eu dei muito poucas dicas sobre como vocês devem escrever as
soluções dos exercícios... isso foi de propósito. O nível de detalhe
esperado varia de acordo com o contexto, e até agora vocês só
precisavam de soluções que vocês mesmos entendessem e tivessem certeza
de cada passo, e que os colegas de vocês entendessem quando vocês
fossem discutir com eles.

Leia a ``dica 7'' daqui:

\msk

\url{http://angg.twu.net/LATEX/material-para-GA.pdf\#page=5}

\msk

Aliás, leia as páginas 4 e 5 inteiras.

\bsk

Depois leia este texto que mandei pras turmas de C2 depois da P1 do
semestre passado:

\msk

\url{http://angg.twu.net/LATEX/2020-1-C2-P1.pdf\#page=10}




\newpage

% «exercicio-5»  (to ".exercicio-5")
% (c2m202escadasp 9 "exercicio-5")
% (c2m202escadas    "exercicio-5")

{\bf Exercício 5.}

\ssk

Seja $F(b) = \Intx{0}{b}{f(x)}$.

\msk

a) Calcule $F(0)$, $F(0.5)$, $F(1)$, $F(1.5)$, $\ldots$, $F(6)$ e
represente os valores que você obteve num gráfico. No gráfico à
direita abaixo eu representei os pontos $(0,F(0))$, $(1,F(1))$ e
$(2,F(2))$ --- faça os outros.

$$
 f(x) \;\; = \;\;
 \vcenter{\hbox{%
 \unitlength=10pt
 \beginpicture(0,-2)(6,5)
   \pictgrid%
   \pictpiecewise{(0,3)--(1,3)o (1,4)c (1,2)o--(3,2)c (3,-1)o--(6,-1)}%
   \pictaxes%
 \end{picture}%
 }}
 \qquad
 F(b) \;\; = \;\;
 \vcenter{\hbox{%
 \unitlength=10pt
 \beginpicture(0,0)(6,7)
   \pictgrid%
   \pictpiecewise{(0,0)c (1,3)c (2,5)c}%
   \pictaxes%
 \end{picture}%
 }}
$$

\bsk

b) Represente graficamente $\Intx{0}{1.5}{f(x)} - \Intx{0}{0.5}{f(x)}$
como

\phantom{b) }uma área no gráfico da $f$.

c) Represente graficamente $F(1.5) - F(0.5)$ no gráfico da $F$.

\newpage

{\bf Exercício 5 (cont.)}

\ssk

Nos itens (b) e (c) do slide anterior nós vimos que uma diferença
%
$$F(d) - F(c) = \Intx{0}{d}{f(x)} - \Intx{0}{c}{f(x)}$$

pode ser interpretada tanto como uma área no gráfico à esquerda quanto
como uma diferença de altura no gráfico à direita. Nos próximos itens
você vai ter que usar essa dupla interpretação em todo lugar.

\msk

d) Verifique que $F(1.3)-F(1.2)$, $F(1.4)-F(1.3)$, $F(1.5)-F(1.4)$ e
$F(1.6)-F(1.5)$ são retângulos com a mesma área --- e verifique que
isto quer dizer que os pontos $(1.2, F(1.2))$, $(1.3, F(1.3))$, $(1.4,
F(1.4))$, $(1.5, F(1.5))$ e $(1.6, F(1.6))$ estão na mesma reta. Qual
é base e a altura de cada um desses retângulos? Qual é o coeficiente
angular dessa reta?

e) Faça o mesmo para estes valores de $x$: 3.2, 3.3, 3.4 e 3.5 --- as
alturas e o coeficiente angular vão mudar.


\newpage

{\bf Exercício 5 (cont.)}

\ssk

A $F(b)$ vai ser contínua, e o gráfico dela vai ser formado por três
segmentos de reta. Pense sozinho em porque isto é verdade --- nós
vamos demonstrar isto em breve.

\msk

f) Complete o gráfico da $F(b)$ (do item a).

g) Em que pontos a $F(b)$ é derivável?

h) Em que pontos a $F(b)$ não é derivável?

i) Seja $g(b) = \frac{d}{db} F(b)$. Faça o gráfico da $g(b)$.

j) Qual é o domínio da $g(b)$?

k) Em que pontos $f(x)$ e $g(b)$ coincidem?


\newpage

% «exercicio-6»  (to ".exercicio-6")
% (c2m202escadasp 12 "exercicio-6")
% (c2m202escadas     "exercicio-6")

{\bf Exercício 6.}

\ssk

Agora que você entendeu a relação entre a $f(x)$ e $F(b) =
\Intx{0}{b}{f(x)}$ num caso específico você vai tentar fazer um outro
caso. Seja $f(x)$ a função à esquerda abaixo, e desenhe o gráfico da
$F(b)$ --- o ``gráfico da integral de $f(x)$'' --- à direita. Obs:
depois que a gente tem prática dá pra resolver problemas assim
sem nenhum erro em poucos segundos! Sério!!!
%
$$
 f(x) \;\; = \;\;
 \vcenter{\hbox{%
 \unitlength=10pt
 \beginpicture(0,-3)(8,4)
   \pictgrid%
   \pictpiecewise{(0,2)--(1,2)o (1,1)c--(2,1)o (2,0)c--(3,0)o
                  (3,-1)c--(4,-1)o (4,-2)o--(6,-2)o (6,-1)c--(7,-1)c (7,0)o--(8,0)}%
   \pictaxes%
 \end{picture}%
 }}
 \qquad
 F(b) = 
 \Intx{0}{b}{f(x)} =
 \;\;
 \vcenter{\hbox{%
 \unitlength=10pt
 \beginpicture(0,-3)(8,4)
   \pictgrid%
   %\pictpiecewise{(0,0)c (1,3)c (2,5)c}%
   \pictaxes%
 \end{picture}%
 }}
$$


\newpage

% «exercicio-7»  (to ".exercicio-7")
% (c2m202escadasp 11 "exercicio-7")
% (c2m202escadas     "exercicio-7")

{\bf Exercício 7.}

\ssk

No exercício 6 você fez o gráfico de $F(b) = \Intx{0}{b}{f(x)}$.

b) Agora faça o gráfico de $G(b) = \Intx{\ColorRed{1}}{b}{f(x)}$,

c) ...e o gráfico de $H(b) = \Intx{\ColorRed{2}}{b}{f(x)}$.



\msk

Você provavelmente desenhou o gráfico da sua $G(b)$ como se ela só
estivesse definida a partir de $b=1$, e o gráfico da $H(b)$ como se
ela só estivesse definida a partir de $b=2$. Isso pode ser melhorado.
Dê uma olhada na página 10 das notas do Pierluigi, onde ele
\ColorRed{define} ``integrais com extremos na ordem inversa'' por esta
regra aqui:
%
$$\Intx{b}{a}{f(x)} = - \Intx{a}{b}{f(x)}$$

% (find-pierluigipage 10    "integral com extremos na ordem inversa")
% (find-pierluigitext 10    "integral com extremos na ordem inversa")

d) Calcule $G(0.5)$.

\newpage

{\bf Exercício 7 (cont.)}

\ssk

e) Agora que você aprendeu a calcular $G(b)$ para $b<1$ usando o
truque da ``integral com extremos na ordem inversa'' faça uma versão
melhorada do gráfico do item (b) na qual o seu gráfico da $G(b)$
inclua os valores de $G(b)$ para $b∈[0,1]$.

f) Faça o mesmo para o item (c): faça uma versão melhorada do gráfico
da $H(b)$.

\msk

g) (Importantíssimo!) Verifique que tanto $F(b)$ quanto $G(b)$ e
$H(b)$ são funções contínuas que obedecem
%
$$F'(b) = G'(b) = H'(b) = f(b)$$
%
em todos os pontos em que essas derivadas fazem sentido --- que são
exatamente os pontos em que a $f(b)$ é contínua.

\newpage

{\bf Primitivas}

\ssk

As funções $F(x)$, $G(x)$ e $H(x)$ que você obteve no exercícios 6 e 7
são ``\ColorRed{primitivas}'' da função $f(x)$. A definição usual de
primitiva que você vai encontrar nos livros é esta aqui:

\begin{quote}
Seja $f:[a,b]→\R$ uma função contínua. Dizemos que uma função
$F:[a,b]→\R$ é uma {\sl primitiva de $f$} quando
$∀x∈(a,b).\;F'(x)=f(x)$.
\end{quote}

\newpage

{\bf Primitivas (2)}

\ssk

Nós vamos usar uma definição um pouco mais complicada de primitiva...
esta aqui:

\begin{quote}
Seja $f:[a,b]→\R$, seja $P$ uma partição de $[a,b]$, e digamos que a
função $f$ seja contínua em cada intervalo aberto $(a_i,b_i)$ da
partição --- ou seja, $f$ não precisa ser contínua nos pontos de $P$.
Uma função $F:[a,b]→\R$ é uma {\sl primitiva de $f$} se: 1) $F$ é
contínua em $[a,b]$, 2) $F$ é derivável em todos os pontos de
$[a,b]∖P$, 3) $∀x∈[a,b]∖P. \; F'(x)=f(x)$.
\end{quote}

\msk

{\bf Exercício 8.}

Verifique que as funções $F(x)$, $G(x)$ e $H(x)$ dos exercícios 6 e 7
são primitivas para a função $f(x)$ do slide 12. Dica: você vai ter
que escolher $[a,b]$ e $P$ da forma certa.

\newpage


% «primitivas-como-usar»  (to ".primitivas-como-usar")
% (c2m202escadasp 17 "primitivas-como-usar")
% (c2m202escadas     "primitivas-como-usar")

{\bf Primitivas: como usar}

\ssk

Se a função $F:[a,b]→\R$ é uma primitiva da função $f:[a,b]→\R$ e
$c,d∈[a,b]$, então podemos usar a $F$ pra calcular integrais da $f$:
%
$$\Intx{c}{d}{f(x)} = F(d) - F(c)$$

Isto é exatamente o que você fez nos exercício 5b e 5c, mas lá
estávamos olhando pra um caso particular muito simples... Isto vale em
geral, mesmo quando a nossa função $f(x)$ não é uma função escada...

...por exemplo, isto vale pra ``nossa função preferida'' das primeiras
aulas, $f(x) = 4 - (x-2)^2$, cujo gráfico é um pedaço de parábola.

\newpage

{\bf Exercício 9.}

Sejam $f(x) = 4 - (x-2)^2$ e $F(x) = -\frac13 x^3 + 2x^2$.

a) Verifique que a função $F$ é uma primitiva para a $f$.

b) Verifique que a função $G(x) = F(x) + 200$ é uma outra primitiva
para a $f$.

c) Calcule $\Intx{0}{4}{f(x)}$ usando isto aqui:
%
$$\Intx{0}{4}{f(x)} = F(4) - F(0)$$

d) Relembre os modos de obter aproximações para integrais que você
aprendeu muitas aulas atrás... por exemplo, o método dos trapézios dá
resultados bastante bons. Compare os resultados dessas aproximações
com o resultado exato da área $\Intx{0}{4}{f(x)}$ que você acabou de
obter.




% (find-pierluigipage 10 "teorema fundamental do cálculo integral")
% (find-pierluigitext 10 "teorema fundamental do cálculo integral")
% (find-pierluigipage 12 "Teorema 15 (Teorema fundamental do cálculo integral)")
% (find-pierluigitext 12 "Teorema 15 (Teorema fundamental do cálculo integral)")




\msk



%\printbibliography

\GenericWarning{Success:}{Success!!!}  % Used by `M-x cv'

\end{document}

%  ____  _             _         
% |  _ \(_)_   ___   _(_)_______ 
% | | | | \ \ / / | | | |_  / _ \
% | |_| | |\ V /| |_| | |/ /  __/
% |____// | \_/  \__,_|_/___\___|
%     |__/                       
%
% «djvuize»  (to ".djvuize")
% (find-LATEXgrep "grep --color -nH --null -e djvuize 2020-1*.tex")

 (eepitch-shell)
 (eepitch-kill)
 (eepitch-shell)
# (find-fline "~/2020.2-C2/")
# (find-fline "~/LATEX/2020-2-C2/")
# (find-fline "~/bin/djvuize")

cd /tmp/
for i in *.jpg; do echo f $(basename $i .jpg); done

f () { rm -fv $1.png $1.pdf; djvuize $1.pdf }
f () { rm -fv $1.png $1.pdf; djvuize WHITEBOARDOPTS="-m 1.0" $1.pdf; xpdf $1.pdf }
f () { rm -fv $1.png $1.pdf; djvuize WHITEBOARDOPTS="-m 0.5" $1.pdf; xpdf $1.pdf }
f () { rm -fv $1.png $1.pdf; djvuize WHITEBOARDOPTS="-m 0.25" $1.pdf; xpdf $1.pdf }
f () { cp -fv $1.png $1.pdf       ~/2020.2-C2/
       cp -fv        $1.pdf ~/LATEX/2020-2-C2/
       cat <<%%%
% (find-latexscan-links "C2" "$1")
%%%
}

f 20201213_area_em_funcao_de_theta
f 20201213_area_em_funcao_de_x
f 20201213_area_fatias_pizza



%  __  __       _        
% |  \/  | __ _| | _____ 
% | |\/| |/ _` | |/ / _ \
% | |  | | (_| |   <  __/
% |_|  |_|\__,_|_|\_\___|
%                        
% <make>

 (eepitch-shell)
 (eepitch-kill)
 (eepitch-shell)
# (find-LATEXfile "2019planar-has-1.mk")
make -f 2019.mk STEM=2020-2-C2-escadas veryclean
make -f 2019.mk STEM=2020-2-C2-escadas pdf

% Local Variables:
% coding: utf-8-unix
% ee-tla: "c2m202escadas"
% End:
