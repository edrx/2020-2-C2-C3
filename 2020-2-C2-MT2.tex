% (find-LATEX "2020-2-C2-MT2.tex")
% (defun c () (interactive) (find-LATEXsh "lualatex -record 2020-2-C2-MT2.tex" :end))
% (defun C () (interactive) (find-LATEXsh "lualatex 2020-2-C2-MT2.tex" "Success!!!"))
% (defun D () (interactive) (find-pdf-page      "~/LATEX/2020-2-C2-MT2.pdf"))
% (defun d () (interactive) (find-pdftools-page "~/LATEX/2020-2-C2-MT2.pdf"))
% (defun e () (interactive) (find-LATEX "2020-2-C2-MT2.tex"))
% (defun o () (interactive) (find-LATEX "2020-2-C2-MT2.tex"))
% (defun u () (interactive) (find-latex-upload-links "2020-2-C2-MT2"))
% (defun v () (interactive) (find-2a '(e) '(d)))
% (defun d0 () (interactive) (find-ebuffer "2020-2-C2-MT2.pdf"))
% (defun cv () (interactive) (C) (ee-kill-this-buffer) (v) (g))
%          (code-eec-LATEX "2020-2-C2-MT2")
% (find-pdf-page   "~/LATEX/2020-2-C2-MT2.pdf")
% (find-sh0 "cp -v  ~/LATEX/2020-2-C2-MT2.pdf /tmp/")
% (find-sh0 "cp -v  ~/LATEX/2020-2-C2-MT2.pdf /tmp/pen/")
%     (find-xournalpp "/tmp/2020-2-C2-MT2.pdf")
%   file:///home/edrx/LATEX/2020-2-C2-MT2.pdf
%               file:///tmp/2020-2-C2-MT2.pdf
%           file:///tmp/pen/2020-2-C2-MT2.pdf
% http://angg.twu.net/LATEX/2020-2-C2-MT2.pdf
% (find-LATEX "2019.mk")
% (find-CN-aula-links "2020-2-C2-MT2" "2" "c2m202mt2" "c2mt2")
%
% Video:
% (find-ssr-links "c2m202mt2" "2020-2-C2-MT2" "{naoexiste}")
% (code-video     "c2m202mt2video" "$S/http/angg.twu.net/eev-videos/2020-2-C2-MT2.mp4")
% (find-c2m202mt2video "0:00")

% «.defs»	(to "defs")
% «.title»	(to "title")
% «.regras»	(to "regras")
% «.dicas»	(to "dicas")
% «.thomas»	(to "thomas")
% «.questao»	(to "questao")
%
% «.djvuize»	(to "djvuize")

\documentclass[oneside,12pt]{article}
\usepackage[colorlinks,citecolor=DarkRed,urlcolor=DarkRed]{hyperref} % (find-es "tex" "hyperref")
\usepackage{amsmath}
\usepackage{amsfonts}
\usepackage{amssymb}
\usepackage{pict2e}
\usepackage[x11names,svgnames]{xcolor} % (find-es "tex" "xcolor")
\usepackage{colorweb}                  % (find-es "tex" "colorweb")
%\usepackage{tikz}
%
% (find-dn6 "preamble6.lua" "preamble0")
%\usepackage{proof}   % For derivation trees ("%:" lines)
%\input diagxy        % For 2D diagrams ("%D" lines)
%\xyoption{curve}     % For the ".curve=" feature in 2D diagrams
%
\usepackage{edrx15}               % (find-LATEX "edrx15.sty")
\input edrxaccents.tex            % (find-LATEX "edrxaccents.tex")
\input edrxchars.tex              % (find-LATEX "edrxchars.tex")
\input edrxheadfoot.tex           % (find-LATEX "edrxheadfoot.tex")
\input edrxgac2.tex               % (find-LATEX "edrxgac2.tex")
%
%\usepackage[backend=biber,
%   style=alphabetic]{biblatex}            % (find-es "tex" "biber")
%\addbibresource{catsem-slides.bib}        % (find-LATEX "catsem-slides.bib")
%
% (find-es "tex" "geometry")
\usepackage[a6paper, landscape,
            top=1.5cm, bottom=.25cm, left=1cm, right=1cm, includefoot
           ]{geometry}
%
\begin{document}

%\catcode`\^^J=10
%\directlua{dofile "dednat6load.lua"}  % (find-LATEX "dednat6load.lua")

% %L dofile "edrxtikz.lua"  -- (find-LATEX "edrxtikz.lua")
% %L dofile "edrxpict.lua"  -- (find-LATEX "edrxpict.lua")
% \pu

% «defs»  (to ".defs")
% (find-LATEX "edrx15.sty" "colors-2019")
\long\def\ColorRed   #1{{\color{Red1}#1}}
\long\def\ColorViolet#1{{\color{MagentaVioletLight}#1}}
\long\def\ColorViolet#1{{\color{Violet!50!black}#1}}
\long\def\ColorGreen #1{{\color{SpringDarkHard}#1}}
\long\def\ColorGreen #1{{\color{SpringGreenDark}#1}}
\long\def\ColorGreen #1{{\color{SpringGreen4}#1}}
\long\def\ColorGray  #1{{\color{GrayLight}#1}}
\long\def\ColorGray  #1{{\color{black!30!white}#1}}
\long\def\ColorBrown #1{{\color{Brown}#1}}
\long\def\ColorBrown #1{{\color{brown}#1}}
\long\def\ColorOrange#1{{\color{orange}#1}}

\long\def\ColorShort #1{{\color{SpringGreen4}#1}}
\long\def\ColorLong  #1{{\color{Red1}#1}}

\def\frown{\ensuremath{{=}{(}}}
\def\True {\mathbf{V}}
\def\False{\mathbf{F}}
\def\D    {\displaystyle}

\def\drafturl{http://angg.twu.net/LATEX/2020-2-C2.pdf}
\def\drafturl{http://angg.twu.net/2020.2-C2.html}
\def\draftfooter{\tiny \href{\drafturl}{\jobname{}} \ColorBrown{\shorttoday{} \hours}}



%  _____ _ _   _                               
% |_   _(_) |_| | ___   _ __   __ _  __ _  ___ 
%   | | | | __| |/ _ \ | '_ \ / _` |/ _` |/ _ \
%   | | | | |_| |  __/ | |_) | (_| | (_| |  __/
%   |_| |_|\__|_|\___| | .__/ \__,_|\__, |\___|
%                      |_|          |___/      
%
% «title»  (to ".title")
% (c2m202mt2p 1 "title")
% (c2m202mt2    "title")

\thispagestyle{empty}

\begin{center}

\vspace*{1.2cm}

{\bf \Large Cálculo 2 - 2020.2}

\bsk

Mini-teste 2.

\bsk

Eduardo Ochs - RCN/PURO/UFF

\url{http://angg.twu.net/2020.2-C2.html}

\end{center}

\newpage

% (c2mt1)

%  ____                          
% |  _ \ ___  __ _ _ __ __ _ ___ 
% | |_) / _ \/ _` | '__/ _` / __|
% |  _ <  __/ (_| | | | (_| \__ \
% |_| \_\___|\__, |_|  \__,_|___/
%            |___/               
%
% «regras»  (to ".regras")
% (c2m202mt1p 2 "regras")
% (c2m202mt1    "regras")

{\bf Regras para o mini-teste}

% (c2m201mt1p 7 "miniteste-regras")
% (c2m201mt1    "miniteste-regras")

As questões do mini-teste serão disponibilizadas às 16:00 da
sexta-feira 9/abril/2021 e você deverá entregar as respostas
\ColorRed{escritas à mão} até as 20:00 do sábado 10/abril/2021 na
plataforma Classroom; desenhos feitos no computador serão
\ColorRed{ignorados}.

Se o Classroom der algum problema mande também para este endereço de
e-mail:

\ssk

\ColorRed{eduardoochs@gmail.com}

\ssk

Mini-testes entregues após este horário não serão considerados.

Durante as 24 horas do mini-teste nem o professor nem o monitor
responderão perguntas sobre os assuntos do mini-teste mas você pode
discutir com os seus colegas --- inclusive no grupo da turma.

Este mini-teste vale 0.5 pontos extras na P1.


\newpage

% «dicas»  (to ".dicas")
% (c2m202mt2p 3 "dicas")
% (c2m202mt2    "dicas")
% (find-LATEX "material-para-GA.tex" "dicas")
% (find-LATEX "material-para-GA.tex" "dicas" "7) Uma solução bem escrita")
% (c2m201p1p 10 "comentario-telegram")
% (c2m201p1     "comentario-telegram")

{\bf Dicas}

\ssk

Leia a ``dica 7'' daqui:

\ssk

% http://angg.twu.net/LATEX/material-para-GA.pdf#page=5
\url{http://angg.twu.net/LATEX/material-para-GA.pdf\#page=5}

\bsk

Além disso revise {\bf MUITO} bem as suas resposta!

Leia esta bronca que eu dei na turma de C2 do semestre passado:

\ssk

% http://angg.twu.net/LATEX/2020-1-C2-P1.pdf#page=10
\url{http://angg.twu.net/LATEX/2020-1-C2-P1.pdf\#page=10}

\newpage

O mini-teste vai ter uma questão só --- uma integral que vocês vão ter
que resolver fazendo duas substituições ``comuns'', ou seja, que não
são nem as que usamos pra integrar potências de senos e cossenos e nem
as da subtituição trigonométrica... vocês vão ter que escrever as
contas bem passo a passo pondo os bloquinhos de anotações à direita e
depois vocês vão ter que verificar o resultado.

\msk

Tem um exemplo de integração com anotações à direita aqui:

\ssk

% http://angg.twu.net/LATEX/2020-2-C2-int-subst.pdf#page=20
\url{http://angg.twu.net/LATEX/2020-2-C2-int-subst.pdf\#page=20}



\newpage

% «thomas»  (to ".thomas")
% http://angg.twu.net/2020.2-C2/thomas_secoes_5.5_e_5.6.pdf
% (code-pdf-page "thomas5556" "$S/http/angg.twu.net/2020.2-C2/thomas_secoes_5.5_e_5.6.pdf")
% (code-pdf-text "thomas5556" "$S/http/angg.twu.net/2020.2-C2/thomas_secoes_5.5_e_5.6.pdf")
% (find-thomas5556page)
% (find-thomas5556text)
% (find-thomas5556page 9 "12.")
% (find-thomas5556text 9 "12.")
% (find-thomas5556page 5 "exploratory tool")
% (find-thomas5556text 5 "exploratory tool")

{\bf Links pra trechos de livros}

Um bom lugar pra estudar é o capítulo 5 do Thomas.

Eu pus uma cópia dele aqui:

\ssk

{\footnotesize

% http://angg.twu.net/2020.2-C2/thomas_secoes_5.5_e_5.6.pdf
\url{http://angg.twu.net/2020.2-C2/thomas_secoes_5.5_e_5.6.pdf}

}

\ssk

mas vou deletar esse link daqui a poucos dias pra não correr risco

de receber ameaças da editora!!!


\newpage

% «questao»  (to ".questao")
% (c2m202mt2p 4 "questao")
% (c2m202mt2    "questao")

{\bf O miniteste}

Calcule

$$\intx {\frac{(2\sqrt{x} + 3)^{10}}{\sqrt{x}}} $$

fazendo duas mudanças de variável e teste a sua resposta.



%\printbibliography

\GenericWarning{Success:}{Success!!!}  % Used by `M-x cv'

\end{document}

%  ____  _             _         
% |  _ \(_)_   ___   _(_)_______ 
% | | | | \ \ / / | | | |_  / _ \
% | |_| | |\ V /| |_| | |/ /  __/
% |____// | \_/  \__,_|_/___\___|
%     |__/                       
%
% «djvuize»  (to ".djvuize")
% (find-LATEXgrep "grep --color -nH --null -e djvuize 2020-1*.tex")

 (eepitch-shell)
 (eepitch-kill)
 (eepitch-shell)
# (find-fline "~/2020.2-C2/")
# (find-fline "~/LATEX/2020-2-C2/")
# (find-fline "~/bin/djvuize")

cd /tmp/
for i in *.jpg; do echo f $(basename $i .jpg); done

f () { rm -fv $1.png $1.pdf; djvuize $1.pdf }
f () { rm -fv $1.png $1.pdf; djvuize WHITEBOARDOPTS="-m 1.0" $1.pdf; xpdf $1.pdf }
f () { rm -fv $1.png $1.pdf; djvuize WHITEBOARDOPTS="-m 0.5" $1.pdf; xpdf $1.pdf }
f () { rm -fv $1.png $1.pdf; djvuize WHITEBOARDOPTS="-m 0.25" $1.pdf; xpdf $1.pdf }
f () { cp -fv $1.png $1.pdf       ~/2020.2-C2/
       cp -fv        $1.pdf ~/LATEX/2020-2-C2/
       cat <<%%%
% (find-latexscan-links "C2" "$1")
%%%
}

f 20201213_area_em_funcao_de_theta
f 20201213_area_em_funcao_de_x
f 20201213_area_fatias_pizza



%  __  __       _        
% |  \/  | __ _| | _____ 
% | |\/| |/ _` | |/ / _ \
% | |  | | (_| |   <  __/
% |_|  |_|\__,_|_|\_\___|
%                        
% <make>

 (eepitch-shell)
 (eepitch-kill)
 (eepitch-shell)
# (find-LATEXfile "2019planar-has-1.mk")
make -f 2019.mk STEM=2020-2-C2-MT2 veryclean
make -f 2019.mk STEM=2020-2-C2-MT2 pdf

% Local Variables:
% coding: utf-8-unix
% ee-tla: "c2m202mt2"
% End:
