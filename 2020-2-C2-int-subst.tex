% (find-LATEX "2020-2-C2-int-subst.tex")
% (defun c () (interactive) (find-LATEXsh "lualatex -record 2020-2-C2-int-subst.tex" :end))
% (defun C () (interactive) (find-LATEXsh "lualatex 2020-2-C2-int-subst.tex" "Success!!!"))
% (defun D () (interactive) (find-pdf-page      "~/LATEX/2020-2-C2-int-subst.pdf"))
% (defun d () (interactive) (find-pdftools-page "~/LATEX/2020-2-C2-int-subst.pdf"))
% (defun e () (interactive) (find-LATEX "2020-2-C2-int-subst.tex"))
% (defun o () (interactive) (find-LATEX "2020-2-C2-int-subst.tex"))
% (defun u () (interactive) (find-latex-upload-links "2020-2-C2-int-subst"))
% (defun v () (interactive) (find-2a '(e) '(d)))
% (defun d0 () (interactive) (find-ebuffer "2020-2-C2-int-subst.pdf"))
% (defun cv () (interactive) (C) (ee-kill-this-buffer) (v) (g))
%          (code-eec-LATEX "2020-2-C2-int-subst")
% (find-pdf-page   "~/LATEX/2020-2-C2-int-subst.pdf")
% (find-sh0 "cp -v  ~/LATEX/2020-2-C2-int-subst.pdf /tmp/")
% (find-sh0 "cp -v  ~/LATEX/2020-2-C2-int-subst.pdf /tmp/pen/")
%     (find-xournalpp "/tmp/2020-2-C2-int-subst.pdf")
%   file:///home/edrx/LATEX/2020-2-C2-int-subst.pdf
%               file:///tmp/2020-2-C2-int-subst.pdf
%           file:///tmp/pen/2020-2-C2-int-subst.pdf
% http://angg.twu.net/LATEX/2020-2-C2-int-subst.pdf
% (find-LATEX "2019.mk")
% (find-CN-aula-links "2020-2-C2-int-subst" "2" "c2m202is" "c2is")
%
% Video:
% (find-ssr-links "c2m202is" "2020-2-C2-int-subst")
% (code-video     "c2m202isvideo" "$S/http/angg.twu.net/eev-videos/2020-2-C2-int-subst.mp4")
% (find-c2m202isvideo "0:00")

% «.defs»			(to "defs")
% «.subst-defs»			(to "subst-defs")
% «.title»			(to "title")
% «.exercicio-1»		(to "exercicio-1")
% «.dicas-subst»		(to "dicas-subst")
% «.dicas-subst-2»		(to "dicas-subst-2")
% «.exemplo-gamb»		(to "exemplo-gamb")
% «.gamb-livros»		(to "gamb-livros")
% «.exercicio-2»		(to "exercicio-2")
% «.exemplo-1»			(to "exemplo-1")
% «.exemplo-1-verif»		(to "exemplo-1-verif")
% «.anotacoes-a-direita»	(to "anotacoes-a-direita")
% «.exemplo-2»			(to "exemplo-2")
% «.exercicio-3»		(to "exercicio-3")
% «.subst-trig-exemplo»		(to "subst-trig-exemplo")
%
% «.djvuize»			(to "djvuize")

\documentclass[oneside,12pt]{article}
\usepackage[colorlinks,citecolor=DarkRed,urlcolor=DarkRed]{hyperref} % (find-es "tex" "hyperref")
\usepackage{amsmath}
\usepackage{amsfonts}
\usepackage{amssymb}
\usepackage{pict2e}
\usepackage[x11names,svgnames]{xcolor} % (find-es "tex" "xcolor")
\usepackage{colorweb}                  % (find-es "tex" "colorweb")
%\usepackage{tikz}
%
% (find-dn6 "preamble6.lua" "preamble0")
%\usepackage{proof}   % For derivation trees ("%:" lines)
%\input diagxy        % For 2D diagrams ("%D" lines)
%\xyoption{curve}     % For the ".curve=" feature in 2D diagrams
%
\usepackage{edrx15}               % (find-LATEX "edrx15.sty")
\input edrxaccents.tex            % (find-LATEX "edrxaccents.tex")
\input edrxchars.tex              % (find-LATEX "edrxchars.tex")
\input edrxheadfoot.tex           % (find-LATEX "edrxheadfoot.tex")
\input edrxgac2.tex               % (find-LATEX "edrxgac2.tex")
%
%\usepackage[backend=biber,
%   style=alphabetic]{biblatex}            % (find-es "tex" "biber")
%\addbibresource{catsem-slides.bib}        % (find-LATEX "catsem-slides.bib")
%
% (find-es "tex" "geometry")
\usepackage[a6paper, landscape,
            top=1.5cm, bottom=.25cm, left=1cm, right=1cm, includefoot
           ]{geometry}
%
\begin{document}

%\catcode`\^^J=10
%\directlua{dofile "dednat6load.lua"}  % (find-LATEX "dednat6load.lua")

% %L dofile "edrxtikz.lua"  -- (find-LATEX "edrxtikz.lua")
% %L dofile "edrxpict.lua"  -- (find-LATEX "edrxpict.lua")
% \pu

% «defs»  (to ".defs")
% (find-LATEX "edrx15.sty" "colors-2019")
\long\def\ColorRed   #1{{\color{Red1}#1}}
\long\def\ColorViolet#1{{\color{MagentaVioletLight}#1}}
\long\def\ColorViolet#1{{\color{Violet!50!black}#1}}
\long\def\ColorGreen #1{{\color{SpringDarkHard}#1}}
\long\def\ColorGreen #1{{\color{SpringGreenDark}#1}}
\long\def\ColorGreen #1{{\color{SpringGreen4}#1}}
\long\def\ColorGray  #1{{\color{GrayLight}#1}}
\long\def\ColorGray  #1{{\color{black!30!white}#1}}
\long\def\ColorBrown #1{{\color{Brown}#1}}
\long\def\ColorBrown #1{{\color{brown}#1}}
\long\def\ColorOrange#1{{\color{orange}#1}}

\long\def\ColorShort #1{{\color{SpringGreen4}#1}}
\long\def\ColorLong  #1{{\color{Red1}#1}}

\def\frown{\ensuremath{{=}{(}}}
\def\True {\mathbf{V}}
\def\False{\mathbf{F}}
\def\D    {\displaystyle}

\def\drafturl{http://angg.twu.net/LATEX/2020-2-C2.pdf}
\def\drafturl{http://angg.twu.net/2020.2-C2.html}
\def\draftfooter{\tiny \href{\drafturl}{\jobname{}} \ColorBrown{\shorttoday{} \hours}}


% «subst-defs»  (to ".subst-defs")
% (find-LATEX "2020-1-C2-TFC2-2.tex" "subst-defs")

\def\pfo#1{\ensuremath{\mathsf{[#1]}}}
\def\veq{\rotatebox{90}{$=$}}
\def\Rd{\ColorRed}
\def\D{\displaystyle}

% Difference with mathstrut
\def\Difms #1#2#3{\left. \mathstrut #3 \right|_{s=#1}^{s=#2}}
\def\Difmu #1#2#3{\left. \mathstrut #3 \right|_{u=#1}^{u=#2}}
\def\Difmx #1#2#3{\left. \mathstrut #3 \right|_{x=#1}^{x=#2}}
\def\Difmth#1#2#3{\left. \mathstrut #3 \right|_{θ=#1}^{θ=#2}}

\def\iequationbox#1#2{
    \left(
    \begin{array}{rcl}
    \D{ #1 } &=& \D{ #2 } \\
    \end{array}
    \right)
  }
\def\isubstbox#1#2#3#4#5{{
    \def\veq{\rotatebox{90}{$=$}}
    \def\ph{\phantom}
    \left(
    \begin{array}{rcl}
    \D{ #1 } &=& \D{ #2 } \\
    {\veq#3} \\
    \D{ #4 } &=& \D{ #5 } \\
    \end{array}
    \right)
  }}
\def\isubstboxT#1#2#3#4#5#6{{
    \def\veq{\rotatebox{90}{$=$}}
    \def\ph{\phantom}
    \left(
    \begin{array}{rcl}
    \multicolumn{3}{l}{\text{#6}} \\%[5pt]
    \D{ #1 } &=& \D{ #2 } \\
    {\veq#3} \\
    \D{ #4 } &=& \D{ #5 } \\
    \end{array}
    \right)
  }}
\def\isubstboxTT#1#2#3#4#5#6#7{{
    \def\veq{\rotatebox{90}{$=$}}
    \def\ph{\phantom}
    \left(
    \begin{array}{rcl}
    \multicolumn{3}{l}{\text{#6}} \\%[5pt]
    \D{ #1 } &=& \D{ #2 } \\
    {\veq#3} \\
    \D{ #4 } &=& \D{ #5 } \\
    \multicolumn{3}{l}{\text{#7}} \\%[5pt]
    \end{array}
    \right)
  }}

% Definição das fórmulas para integração por substituição.
% Algumas são pmatrizes 3x3 usando isubstbox.

\def\TFCtwo{
  \iequationbox {\Intx{a}{b}{F'(x)}}
                {\Difmx{a}{b}{F(x)}}
}
\def\TFCtwoI{
  \iequationbox {\intx{F'(x)}}
                {F(x)}
}

\def\Sone{
  \isubstbox
    {\Difmx{a}{b}{f(g(x))}}  {\Intx{a}{b}{f'(g(x))g'(x)}}
    {\ph{mmm}}
    {\Difmu{g(a)}{g(b)}{f(u)}} {\Intu{g(a)}{g(b)}{f'(u)}}
}
\def\SoneI{
  \isubstbox
    {f(g(x))} {\intx{f'(g(x))g'(x)}}
    {\ph{m}}
    {f(u)}    {\intu{f'(u)}}
}

\def\Stwo{
  \isubstboxT
    {\Difmx{a}{b}{F(g(x))}}   {\Intx{a}{b}{f(g(x))g'(x)}}
    {\ph{mmm}}
    {\Difmu{g(a)}{g(b)}{F(u)}}  {\Intu{g(a)}{g(b)}{f(u)}}
    {Se $F'(u)=f(u)$ então:}
}
\def\StwoI{
  \isubstboxT
    {F(g(x))}  {\intx{f(g(x))g'(x)}}
    {\ph{m}}
    {F(u)}     {\intu{f(u)}}
    {Se $F'(u)=f(u)$ então:}
}
\def\StwoI{
  \isubstboxTT
    {F(g(x))}  {\intx{f(g(x))g'(x)}}
    {\ph{m}}
    {F(u)}     {\intu{f(u)}}
    {Se $F'(u)=f(u)$ então:}
    {Obs: $u=g(x)$.}
}

\def\Sthree{
  \iequationbox {\Intx{a}{b}{f(g(x))g'(x)}}
                {\Intu{g(a)}{g(b)}{f(u)}}
}
\def\SthreeI{
  \iequationbox {\intx{f(g(x))g'(x)}}
                {\intu{f(u)}
                 \qquad [u=g(x)]
                }
  % [u=g(x)]
}

\def\Sthree{
  \pmat{
    \D \Intx{a}{b}{f(g(x))g'(x)} \\
    \veq \\
    \D \Intu{g(a)}{g(b)}{f(u)}
  }}

\def\SthreeI{
  \pmat{
    \D \intx{f(g(x))g'(x)} \\
       \veq \\
    \D \intu{f(u)} \\
    \text{Obs: $u=g(x)$.} \\
  }}



\def\Subst#1{\bmat{#1}}




%  _____ _ _   _                               
% |_   _(_) |_| | ___   _ __   __ _  __ _  ___ 
%   | | | | __| |/ _ \ | '_ \ / _` |/ _` |/ _ \
%   | | | | |_| |  __/ | |_) | (_| | (_| |  __/
%   |_| |_|\__|_|\___| | .__/ \__,_|\__, |\___|
%                      |_|          |___/      
%
% «title»  (to ".title")
% (c2m202isp 1 "title")
% (c2m202is    "title")

\thispagestyle{empty}

\begin{center}

\vspace*{1.2cm}

{\bf \Large Cálculo 2 - 2020.2}

\bsk

Aula 14: integração por substituição.

\bsk

Eduardo Ochs - RCN/PURO/UFF

\url{http://angg.twu.net/2020.2-C2.html}

\end{center}

\newpage

Nos slides anteriores nós vimos como demonstrar a fórmula \pfo{S2}, e
{\sl começamos} a ver como usá-la na prática... na verdade a gente
costuma usar uma versão da \pfo{S2} para integrais {\sl indefinidas},
em que a gente primeiro faz as contas de uma forma abreviada, omitindo
todos os limites de integração e depois a gente recoloca eles.

A nossa versão para integrais indefinidas da \pfo{S2} vai ser a
\pfo{S2I} do próximo slide. Repare na linha
%
$$\text{``\Rd{Obs: $u=g(x)$}''}$$
%
no final da \pfo{S2I} --- ela vai ser importante pra evitar
ambiguidades.

\msk

No slide seguinte eu pus a nossa versão para integrais indefinidas da
\pfo{S3}, que eu chamei de \pfo{S3I}.

\newpage

$\begin{array}{rcc}
 \pfo{S2} &=& \Stwo \\
 \\
 \pfo{S2I} &=& \StwoI \\
 \end{array}
$

\newpage

$\begin{array}{rcc}
 \pfo{S3}  &=& \Sthree \\
 \\
 \pfo{S3I} &=& \SthreeI \\
 \end{array}
$


\newpage

% «exercicio-1»  (to ".exercicio-1")
% (c2m202isp 5 "exercicio-1")
% (c2m202is    "exercicio-1")

{\bf Exercício 1.}

Calcule os resultados das substituições abaixo:

\msk

a) $\pfo{S2I} \, [g(x):=3x+4]$

b) $\pfo{S2I} \, [g(x):=3x+4] \, [f(u) := \frac23 \cos u]$

c) $\pfo{S3I} \, [g(x):=3x+4]$

d) $\pfo{S3I} \, [g(x):=3x+4] \, [f(u) := \frac23 \cos u]$

\newpage

% «dicas-subst»  (to ".dicas-subst")
% (c2m202isp 6 "dicas-subst")
% (c2m202is    "dicas-subst")

{\bf Mais dicas sobre a operação `$[:=]$'}

\ssk

Nas duas substituições abaixo a primeira está certa

e a segunda está errada:
%
$$\begin{array}{rll}
  (x + 2 = 5) \, [x:=4] &=& (4 + 2 = 5) \\
  (x + 2 = 5) \, [x:=4] &=& (6 = 5) \\
  \end{array}
$$

O `$=$' depois de uma substituição tem um significado especial: a
pronúncia dele é ``o resultado da substituição à esquerda é a
expressão à direita'', e na segunda linha a gente fez mais coisas além
de só substituir todos os `$x$'s por `4's.

Note que isto aqui está certo:
%
$$\begin{array}{rll}
  (x + 2 = 5) \, [x:=4] &=& (4 + 2 = 5) \\
                        &=& (6 = 5) \\
  \end{array}
$$



\newpage

% «dicas-subst-2»  (to ".dicas-subst-2")
% (c2m202isp 7 "dicas-subst-2")
% (c2m202is    "dicas-subst-2")

{\bf Mais dicas sobre a operação `$[:=]$' (2)}

Aqui a primeira está certa e a segunda está errada...

Na segunda um `$u$' foi substituido por `$e^{2x}$'!!!!!!!! $\;\;\;=\!($
%
$$\scalebox{0.9}{$
  \begin{array}{rcl}
  \SthreeI [g(x):=e^{2x}] & = &
     \pmat{ \D \intx{f(2^{2x})(2e^{2x})} \\
            \veq \\
            \D \intu{f(u)} \\
            \text{Obs: $u=e^{2x}$.} \\
          }
  \\
  \\
  \SthreeI [g(x):=e^{2x}] & = &
     \pmat{ \D \intx{f(2^{2x})(2e^{2x})} \\
            \veq \\
            \D \intu{f(\ColorRed{e^{2x}})} \\
            \text{Obs: $u=e^{2x}$.} \\
          }
  \end{array}
  $}
$$



\newpage

{\bf Mais dicas sobre a operação `$[:=]$' (3)}

No primeiro PDF do curso nós usamos a operação `$[:=]$' para testar
EDOs como $f'(x)=x^4$ em vários ``valores'' de $f$, pra tentar
resolver EDOs por chutar-e-testar... Em
%
$$(f'(x)=x^4)\, [f(x):=x^2] = (2x = x^4)$$
%
na expressão original, $(f'(x)=x^4)$, o símbolo $f$ faz o papel de uma
função qualquer, ou de uma variável cujo valor é uma função; a
substiuição ``$[f(x):=x^2]$'' diz como substituir a $f$ original,
genérica, pela $f$ que tem esta {\sl definição} aqui: $f(x)=x^2$... e
nós já temos bastante prática com obter consequencias de uma definição
como $f(x)=x^2$. Por exemplo:
%
$$\begin{array}{rclcrcl}
  f(200)  &=& 200^2            && f'(x) &=& 2x \\
  f(3u+4) &=& (3u+4)^2         && f'(3u+4) &=& 2(3u+4) \\
  f(42x^3+99) &=& (42x^3+99)^2 && f'(42x^3+99) &=& 2(42x^3+99) \\
  \end{array}
$$


\newpage

% «exemplo-gamb»  (to ".exemplo-gamb")
% (c2m202isp 9 "exemplo-gamb")
% (c2m202is    "exemplo-gamb")

{\bf Exemplo com gambiarras}

Isto aqui é um exemplo de como contas com integração

por substituição costumam ser feitas na prática:
%
$$\begin{array}{l}
  \D \intx{2 \cos(3x+4)} \\[8pt]
  = \;\; \D \intu {2 (\cos u) · \frac13} \\[8pt]
  = \;\; \D \frac23 \intu{\cos u} \\[8pt]
  = \;\; \D \frac23 \sen u \\[8pt]
  = \;\; \D \frac23 \sen (3x+4) \\
  \end{array}
$$

A ``Obs: $u=3x+4$'' costuma ser posta no texto em português

antes ou depois das contas, ou omitida (!!!)...

\newpage

% «gamb-livros»  (to ".gamb-livros")
% (c2m202isp 10 "gamb-livros")
% (c2m202is     "gamb-livros")

{\bf A gambiarra nos livros}

Dê uma olhada na página 165 do Martins/Martins.

Eles pulam {\bf MUITOS} passos, e dizem
%
$$\text{Fazendo $u=g(x)$, $du=g'(x)\,dx$, e substituindo em (A)...}$$

\def\und#1#2{\underbrace{#1}_{#2}}

A idéia é esta aqui:
%
$$\D \int {f( \und{g(x)}{u} ) } \, \und{ \und{g'(x)}{\frac{du}{dx}} \, dx}{du}
  \;\;=\;\;
  \D \int f(u)\,du
$$

mas nós só vamos ver um jeito de dar um significado

\Rd{preciso} pra esse ``$\frac{du}{dx}dx = du$'' em Cálculo 3...

\ssk

(Dê uma olhada também na ``Aula 4'' da Cristiane Hernández).

\newpage

% «exercicio-2»  (to ".exercicio-2")
% (c2m202isp 11 "exercicio-2")
% (c2m202is     "exercicio-2")

{\bf Exercício 2.}

\ssk

a) Faça as gambiarras do slide anterior nesta integral daqui,
%
$$\D \intx {(1 - (\sen x)^2) (\sen x)^3 \cos x}$$

para transformá-la numa integral em $u$. Use $u=\sen x$.

\ssk

b) Resolva a integral em $u$ que você acabou de obter.

(O resultado dela vai ser um polinômio em $u$).

\ssk

c) Junte tudo numa série de igualdades como as do

``exemplo com gambiarras''. No final você deve chegar

num expressão em $x$ sem sinal de integral.

\newpage

\thispagestyle{empty}

\begin{center}

\vspace*{1.2cm}

{\bf \Large Integrais de}

\ssk

{\bf \Large  potências de}

\ssk

{\bf \Large senos e cossenos}

\end{center}


\newpage

% «exemplo-1»  (to ".exemplo-1")
% (c2m202isp 12 "exemplo-1")
% (c2m202is     "exemplo-1")
% (c2m201ipscp 2 "exemplo-1")
% (c2m201ipsc    "exemplo-1")

\def\S{\sen x}
\def\C{\cos x}
\def\D{\displaystyle}
\def\und#1#2{\underbrace{#1}_{#2}}

{\bf Exemplo 1}
%
$$\begin{array}[t]{l}
  \D \intx{(\S)^5 (\C)^3} \\
  \D = \;\; \intx{(\S)^5 (\C)^2 (\C)} \\
  \D = \;\; \intx{(\und{\S}{s})^5 \und{(\C)^2}{1-s^2} \und{(\C)}{\frac{ds}{dx}}} \\
  \D = \;\; \ints{s^5 (1-s^2)} \\
  \D = \;\; \ints{s^5 - s^7} \\
  \D = \;\; \frac{s^6}{6} - \frac{s^8}{8} \\
  \D = \;\; \frac{(\S)^6}{6} - \frac{(\S)^8}{8} \\
  \end{array}
  \qquad
  \begin{array}[t]{c}
  \\ \\
    \bmat{s = \sen x \\
          \frac{ds}{dx} = \cos x \\
          \sen x = s \\
          (\cos x)^2 = 1 - s^2 \\
          \cos x \, dx = ds
    }
  \end{array}
$$

\newpage

% «exemplo-1-verif»  (to ".exemplo-1-verif")
% (c2m202isp 13 "exemplo-1-verif")
% (c2m202is     "exemplo-1-verif")
% (c2m201ipscp 3 "exemplo-1-verif")
% (c2m201ipsc    "exemplo-1-verif")

{\bf Exemplo 1: verificação}

$$\begin{array}[t]{l}
  \D \ddx \left(\frac{(\S)^6}{6} - \frac{(\S)^8}{8}\right) \\
  \D = \frac{6(\S)^5\C}{6} - \frac{8(\S)^7\C}{8} \\
  \D = ((\S)^5 - (\S)^7) \C \\
  \D = ((\S)^5(1-(\S)^2)) \C \\
  \D = (\S)^5(\C)^2 \C \\
  \D = (\S)^5(\C)^3 \\
  \end{array}
$$

\newpage

% «anotacoes-a-direita»  (to ".anotacoes-a-direita")
% (c2m202isp 14 "anotacoes-a-direita")
% (c2m202is     "anotacoes-a-direita")

{\bf As anotações à direita}

\ssk

Note que à direita das contas do exemplo 1 tinha isso aqui:
%
$$\bmat{s = \sen x \\
       \frac{ds}{dx} = \cos x \\
       \sen x = s \\
       (\cos x)^2 = 1 - s^2 \\
       \cos x \, dx = ds
  }
$$

que à primeira vista parece com a operação `$[:=]$',

mas exceto pela primeira linha ele não segue \ColorRed{nenhuma}

das convenções sobre o `$[:=]$' que vimos aqui:

\ssk

% http://angg.twu.net/LATEX/2020-2-C2-intro.pdf#page=6
\url{http://angg.twu.net/LATEX/2020-2-C2-intro.pdf\#page=6}

\ssk

E ainda por cima ele é usado duas vezes, uma pra mudar

de $x$ pra $s$ e outra pra voltar pra $x$!...

\newpage

% «exemplo-2»  (to ".exemplo-2")
% (c2m202isp 15 "exemplo-2")
% (c2m202is     "exemplo-2")
% (c2m201ipscp 4 "exemplo-2")
% (c2m201ipsc    "exemplo-2")

{\bf Exemplo 2}
%
$$\begin{array}[t]{l}
  \D \intx{(\S)^5 (\C)^2} \\
  \D = \;\; \intx{(\S)^4 (\C)^2 \S} \\
  \D = \;\; \intx{(\und{(\S)^2}{1-c^2})^2 \und{(\C)^2}{c^2} \und{(\S)}{-\frac{dc}{dx}}} \\
  \D = \;\; \intc{(1-c^2)^2 c^2} \\
  \D = \;\; \intc{(c^4 - 2c^2 + 1) c^2 (-1)} \\
  \D = \;\; \intc{-c^6 + 2c^4 - c^2} \\
  \D = \;\; \ldots \\
  \end{array}
  \qquad
  \begin{array}[t]{c}
  \\ \\
    \bmat{c = \cos x \\
          \frac{dc}{dx} = - \sen x \\
          \cos x = c \\
          (\sen x)^2 = 1 - c^2 \\
          \sen x \, dx = (-1)\,dc
    }
  \end{array}
$$

\newpage

% «exercicio-3»  (to ".exercicio-3")
% (c2m202isp 16 "exercicio-3")
% (c2m202is     "exercicio-3")
% (c2m201ipscp 5 "exercicio-1")
% (c2m201ipsc    "exercicio-1")

{\bf Exercício 3.}

\ssk

a) Calcule a integral do exemplo 1 -- $\D \intx{(\S)^5 (\C)^3}$ --

usando a substituição $c=\cos x$ ao invés de $s=\sen x$.

\msk

b) Teste o seu resultado.

\bsk

Dica: em algum lugar do teste você vai precisar da identidade

$(\cos x)^2 + (\sen x)^2 = 1$... nós vamos começar a usar

identidades trigonométricas a beça.



\newpage

% «pares-e-impares»  (to ".pares-e-impares")
% (c2m201ipscp 6 "pares-e-impares")
% (c2m201ipsc    "pares-e-impares")

{\bf Dica importante}

\ssk

Pra integrar algo como:
%
$$ \intx {(\S)^α (\C)^β}
$$

Se tanto $α$ quanto $β$ são ímpares as duas subtituições,

$s=\S$ e $c=\C$, funcionam.

\msk

Se só um dos dois é ímpar só uma delas funciona

(não vou dizer qual).

\msk

Se tanto $α$ quanto $β$ são \ColorRed{pares} aí \ColorRed{nenhuma

das duas substituições funciona}, e a gente vai precisar

de técnicas mais avançadas que vamos ver depois.


\newpage

\thispagestyle{empty}

\begin{center}

\vspace*{1.2cm}

{\bf \Large Introdução à}

\ssk

{\bf \Large Substituição}

\ssk

{\bf \Large Trigonométrica}

\end{center}


\newpage

% «subst-trig-exemplo»  (to ".subst-trig-exemplo")
% (c2m202isp 20 "subst-trig-exemplo")
% (c2m202is     "subst-trig-exemplo")

\def\St{\sen θ}
\def\Ct{\cos θ}
\def\Sqs{\sqrt{1-s^2}}

$%\scalebox{0.8}{$
  \begin{array}[c]{l}
  \D \ints {s \Sqs} \\
  = \;\; \D \intth {\St \sqrt{1-(\St)^2} \Ct}  \\
  = \;\; \D \intth {\St \sqrt{(\Ct)^2} \Ct}  \\
  = \;\; \D \intth {\St \Ct \Ct}  \\
  = \;\; \D \intth {(\Ct)^2 \St} \\
  = \;\; \D \intc  {c^2 · (-1)} \\
  = \;\; -\frac13 c^3 \\
  = \;\; -\frac13 (\Ct)^3 \\
  = \;\; -\frac13 (\sqrt{1-(\St)^2})^3 \\
  = \;\; -\frac13 (\sqrt{1-s^2})^3 \\
  \end{array}
  \qquad
  \begin{array}[c]{c}
  \bmat{s = \St \\ \frac{ds}{dθ} = \Ct \\ \Ct \, dθ = ds}
  \\ \\
  \bmat{c = \Ct \\ \frac{dc}{dθ} = - \St \\ \St \, dθ = (-1)·dc \\}
  \end{array}
  %$}
$
\newpage

O exemplo da página anterior usa um monte de técnicas nada óbvias.

A das cinco primeiras linhas dá isso aqui,
%
$$\D \ints {s \Sqs} = \intth {(\Ct)^2 \St}$$

Nós vamos começar aprendendo a fazer algo um pouco mais geral:

vamos aprender a ajustar os `$α$'s, `$β$'s, `$γ$'s e `$δ$'s aqui, 
%
$$\D \ints {s^α (\Sqs)^β} = \intth {(\Ct)^γ (\St)^δ}$$

e vamos aprender a fazer algo parecido para as substituições

$t = \tanθ$ e $z = \secθ$.

\ssk

Eu vou chamar o `$\Sqs$' de ``o \ColorRed{termo malvado}'' da integral

$\ints {s \Sqs}$. É ele que faz essa integral ser muito difícil de

resolver, e queremos nos livrar dele.

\newpage

{\bf Abreviações}

No exemplão do slide 20 as letras $s$, $θ$ e $c$ sempre denotam
variáveis,

exceto nas linhas
%
$$\frac{ds}{dθ} = \cosθ \qquad \text{e} \qquad \frac{dc}{dθ} = -\senθ$$

das anotações, que usam a ``notação de Leibniz'', na qual podemos

definir coisas como ``$s = \senθ$'' e ``$c = \cosθ$'' e aí tratar os
`$s$'s e `$c$'s

como \ColorRed{abreviações} para $\senθ$ e $\cosθ$...

\msk

Essas abreviações dão margem pra muita confusão e fazem as pessoas
cometerem zilhões de erros de conta nos primeiros anos até elas terem
vários anos de prática. Como vocês não têm anos pra praticar, nem têm
bibliotecas super confortáveis em que vocês podem passar centenas de
tardes estudando e discutindo junto com os colegas, nós só vamos usar
essas abreviações em situações controladas.

\newpage

{\bf Algumas identidades trigonométricas}

A minha memória é \ColorRed{péssima}. Eu não consigo decorar
praticamente nenhuma fórmula... mas eu consigo lembrar vagamente como
rededuzir certas fórmulas, e aí eu sempre refaço as demonstrações.

No caso das identidades trigonométricas eu consigo rededuzir elas
bastante rápido usando essas abreviações aqui embaixo à esquerda, e as
fórmulas pra derivadas de produtos e quocientes de funções à direita:
%
$$\begin{array}[c]{rcl}
  f &=& f(x) \\
  g &=& g(x) \\
  s &=& \sen θ \\
  c &=& \cos θ \\
  t &=& \tan θ = \D \frac{\senθ}{\cosθ} = \frac{s}{c} \\[10pt]
  z &=& \sec θ = \D \frac{1}{\cosθ} = \frac{1}{c} \\
  \end{array}
  \qquad
  \begin{array}[c]{rcl}
  \D \frac{d}{dx} \frac{1}{g} &=& \D \frac{g'}{g^2} \\[10pt]
  \D \frac{d}{dx} \frac{f}{g} &=& \D \frac{fg' - f'g}{g^2} \\
  \end{array}
$$

\newpage

{\bf Algumas identidades trigonométricas (2)}

$$\begin{array}{rcl}
  z^2 &=& (\frac 1c)^2 = \frac 1{c^2} = \frac{c^2+s^2}{c^2}
          = \frac{c^2}{c^2} + \frac{s^2}{c^2} \\[2.5pt]
      &=& 1 + (\frac sc)^2 \\[2.5pt]
      &=& 1+t^2 \\
  \end{array}
$$

\bsk

$$\begin{array}{lcl}
  c^2 + s^2 = 1 \\
  c^2 = 1 - s^2  && c=\sqrt{1-s^2} \\
  s^2 = 1 - c^2  && s=\sqrt{1-c^2} \\[5pt]
  z^2 = 1 + t^2  && z=\sqrt{1+t^2} \\
  t^2 = z^2 - 1  && t=\sqrt{z^2-1} \\
  \end{array}
$$

\newpage

{\bf Algumas identidades trigonométricas (3)}

$$\begin{array}{ll}
  \frac{ds}{dθ} = c
    \\[2.5pt]
  \frac{dc}{dθ} = -s
    \\[2.5pt]
  \frac{dt}{dθ} = \frac{d}{dθ} \frac{s}{c} = \frac{s'c - sc'}{c^2}
    = \frac{cc - s·(-s)}{c^2} = \frac{c^2+s^2}{c^2} = \frac{1}{c^2} = z^2
    \\[2.5pt]
  \frac{dz}{dθ} = \frac{d}{dθ} \frac{1}{c} = \frac{-c'}{c^2}
    = \frac{s}{c^2} = \frac{1}{c} \frac{s}{c} = zt \\
  \\
  ds =  \cosθ\,dθ     \\[2.5pt]
  dc = -\senθ\,dθ     \\[2.5pt]
  dt = -(\secθ)^2\,dθ \\[2.5pt]
  dz = \secθ \tanθ \,dθ \\[2.5pt]
  \end{array}
$$


% (c2m201substtrig1p 2 "exemplo-1")
% (c2m201substtrig1    "exemplo-1")





% (find-martinscdipage (+ 10 165) "6.1       Metodo da Substituicao")
% (find-martinscditext (+ 10 165) "6.1       Metodo da Substituicao")
% (find-martinscdipage (+ 10 165) "6.1       Metodo da Substituicao")
% (find-martinscditext (+ 10 165) "6.1       Metodo da Substituicao")
% (c2m201tfc22p 2 "integral-indefinida")
% (c2m201tfc22    "integral-indefinida")
% (c2m201tfc22p 4 "exercicio-1")
% (c2m201tfc22    "exercicio-1")
% (c2m201isubp 2 "exemplo-com-gambs")
% (c2m201isub    "exemplo-com-gambs")


%\printbibliography

\GenericWarning{Success:}{Success!!!}  % Used by `M-x cv'

\end{document}

%  ____  _             _         
% |  _ \(_)_   ___   _(_)_______ 
% | | | | \ \ / / | | | |_  / _ \
% | |_| | |\ V /| |_| | |/ /  __/
% |____// | \_/  \__,_|_/___\___|
%     |__/                       
%
% «djvuize»  (to ".djvuize")
% (find-LATEXgrep "grep --color -nH --null -e djvuize 2020-1*.tex")

 (eepitch-shell)
 (eepitch-kill)
 (eepitch-shell)
# (find-fline "~/2020.2-C2/")
# (find-fline "~/LATEX/2020-2-C2/")
# (find-fline "~/bin/djvuize")

cd /tmp/
for i in *.jpg; do echo f $(basename $i .jpg); done

f () { rm -fv $1.png $1.pdf; djvuize $1.pdf }
f () { rm -fv $1.png $1.pdf; djvuize WHITEBOARDOPTS="-m 1.0" $1.pdf; xpdf $1.pdf }
f () { rm -fv $1.png $1.pdf; djvuize WHITEBOARDOPTS="-m 0.5" $1.pdf; xpdf $1.pdf }
f () { rm -fv $1.png $1.pdf; djvuize WHITEBOARDOPTS="-m 0.25" $1.pdf; xpdf $1.pdf }
f () { cp -fv $1.png $1.pdf       ~/2020.2-C2/
       cp -fv        $1.pdf ~/LATEX/2020-2-C2/
       cat <<%%%
% (find-latexscan-links "C2" "$1")
%%%
}

f 20201213_area_em_funcao_de_theta
f 20201213_area_em_funcao_de_x
f 20201213_area_fatias_pizza



%  __  __       _        
% |  \/  | __ _| | _____ 
% | |\/| |/ _` | |/ / _ \
% | |  | | (_| |   <  __/
% |_|  |_|\__,_|_|\_\___|
%                        
% <make>

 (eepitch-shell)
 (eepitch-kill)
 (eepitch-shell)
# (find-LATEXfile "2019planar-has-1.mk")
make -f 2019.mk STEM=2020-2-C2-int-subst veryclean
make -f 2019.mk STEM=2020-2-C2-int-subst pdf

% Local Variables:
% coding: utf-8-unix
% ee-tla: "c2m202is"
% End:
